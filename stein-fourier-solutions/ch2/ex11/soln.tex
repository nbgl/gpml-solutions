\documentclass[oneside]{article}

\usepackage{amsfonts}
\usepackage{amsmath}
\usepackage{amssymb}
\usepackage{amsthm}
\usepackage{enumitem}
\usepackage{mathtools}

\allowdisplaybreaks

% Notation shortcuts
\newcommand\abs[1]{\left|#1\right|}
\newcommand\defeq{\overset{\mathrm{def}}{=}}
\newcommand*\Laplace{\mathop{}\!\mathbin\bigtriangleup}

\DeclarePairedDelimiter\ceil{\lceil}{\rceil}
\DeclarePairedDelimiter\floor{\lfloor}{\rfloor}

\DeclareMathOperator{\sgn}{sgn}
\DeclareMathOperator{\atantwo}{atan2}

\newcommand\bbC{\mathbb{C}}
\newcommand\bbR{\mathbb{R}}
\newcommand\bbZ{\mathbb{Z}}

\newcommand\calU{\mathcal{U}}
\newcommand\calL{\mathcal{L}}

\newtheorem*{lem}{Lemma}
\newtheorem*{cor}{Corollary}

\begin{document}
  We first show that $f$ is integrable.

  Pick an arbitrary $\varepsilon > 0$. Let $k$ be such that \[
    \int_0^1\abs{f_k(x)-f(x)}\,dx < \frac\varepsilon5 \text{.}
  \] Let $P = \{0 = p_0, \dots, p_N=1\}$ be a partition such that\[
    \calU(P, f_k) - \calL(P, f_k) < \frac\varepsilon5
    \quad\text{and}\quad
    \calU(P, \abs{f-f_k}) - \calL(P, \abs{f-f_k}) < \frac\varepsilon5 \text{.}
  \] $P$ exists because $f_k$ and $\abs{f-f_k}$ are integrable\footnote{
    We can find $P$ by finding $P_1$ and $P_2$ such that
    $\calU(P_1, f_k) - \calL(P_1, f_k) < \varepsilon/5$ and
    $\calU(P_2, g_k) - \calL(P_2, g_k) < \varepsilon/5$, and setting
    $P = P_1 \cup P_2$.
  }. Observe that \[
    0 \leq \calL(P, f) \leq \int_0^1 \abs{f_k(x) - f(x)}\,dx
    \leq \calU(P, f)\text,
  \] implying that \[
    \calU(P, f) \leq \frac{2\varepsilon}{5}\text.
  \]

  For convenience, define $I_j = [p_{j-1}, p_j]$ for $j = 1, \dots, N$. We have
  \begin{align*}
    \sup_{x\in I_j} f(x) &\leq \sup_{x\in I} f_k(x)
    + \sup_{x\in I} \abs{f_k - f(x)}
    \text{, and} \\
    \inf_{x\in I_j} f(x) &\geq \inf_{x\in I} f_k(x)
    - \sup_{x\in I} \abs{f_k - f(x)}
  \end{align*} Then \[
    \sup_{x\in I_j}f(x) - \inf_{x\in I_j} f(x)
    \leq \sup_{x\in I} f_k(x) - \inf_{x\in I} f_k(x)
    + 2\sup_{x\in I} \abs{f_k - f(x)} \text{.}
  \] Hence,\begin{align*}
    &\calU(P, f) - \calL(P, f) \\
    &= \sum_{j=1}^N \abs{I_j}\sup_{x\in I_j}f(x)
    - \sum_{j=1}^N \abs{I_j}\inf_{x\in I_j}f(x) \\
    &\leq \sum_{j=1}^N \abs{I_j}\sup_{x\in I_j}f_k(x)
    - \sum_{j=1}^N \abs{I_j}\inf_{x\in I_j}f_k(x)
    + 2\sum_{j=1}^N \abs{I_j}\sup_{x\in I_j}\abs{f_k(x) - f(x)} \\
    &= \calU(P, f_k) - \calL(P, f_k) + 2\calU(P, \abs{f_k - f}) \\
    &< \varepsilon \text{.}
  \end{align*} Since $\varepsilon > 0$ is arbitrary, $f$ is integrable and its
  Fourier coefficients exist.

  We have\begin{align*}
    \abs{\hat{f}_k(n) - \hat{f}_k(n)}
    &= \abs{\int_0^1f_k(x)e^{-2\pi inx}\,dx - \int_0^1f(x)e^{-2\pi inx}\,dx} \\
    &= \abs{\int_0^1\left(f_k(x) - f(x)\right)e^{-2\pi inx}\,dx} \\
    &\leq \int_0^1\abs{\left(f_k(x) - f(x)\right)e^{-2\pi inx}}\,dx \\
    &= \int_0^1\abs{f_k(x) - f(x)}\,dx \text{.}
  \end{align*}

  The last integral is independent of $n$. It tends to $0$ as $k \to \infty$ by
  assumption.\qed
\end{document}
