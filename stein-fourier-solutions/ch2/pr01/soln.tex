\documentclass[oneside]{article}

\usepackage{amsfonts}
\usepackage{amsmath}
\usepackage{amssymb}
\usepackage{amsthm}
\usepackage{enumitem}
\usepackage{mathtools}
\usepackage{cleveref}

\allowdisplaybreaks

% Notation shortcuts
\newcommand\abs[1]{\left|#1\right|}
\newcommand\defeq{\overset{\mathrm{def}}{=}}
\newcommand*\Laplace{\mathop{}\!\mathbin\bigtriangleup}

\DeclarePairedDelimiter\ceil{\lceil}{\rceil}
\DeclarePairedDelimiter\floor{\lfloor}{\rfloor}

\DeclareMathOperator{\sgn}{sgn}
\DeclareMathOperator{\atantwo}{atan2}

\newcommand\bbC{\mathbb{C}}
\newcommand\bbR{\mathbb{R}}
\newcommand\bbQ{\mathbb{Q}}
\newcommand\bbZ{\mathbb{Z}}
\newcommand\calL{\mathcal{L}}
\newcommand\calU{\mathcal{U}}

\newtheorem*{lem}{Lemma}
\newtheorem*{cor}{Corollary}

\renewcommand{\thefootnote}{[\arabic{footnote}]}

\begin{document}
  \begin{enumerate}[label=(\alph*)]
    \item
      We are given that \[
        f(x) = \begin{cases}
          0 &\text{if }x < 0\text{,} \\
          1 &\text{if }x \geq 0 \text{.}
        \end{cases}
      \] Choose a countable dense sequence $\{r_n\}$ in $(0,1]$, for example, an enumeration of $(0,1] \cap \bbQ$. Define $F: [0,1] \to \bbR$ with \[
        F(x) = \sum_{n=1}^\infty \frac1{n^2}f(x-r_n) \text{.}
      \] We want to show that $F$ is integrable and that it is discontinuous at every point of $\{r_n\}$.

      \paragraph{Integrability} It is sufficient to show that $F$ is monotonic. Then $F$ is integrable since it is bounded\footnote{$F$ is bounded above by $\sum_{n=1}^\infty 1/n^2$ and below by $0$ since the codomain of $f$ is $\{0,1\}$.} and monotonic on a closed interval.

      Choose $x$ and $x'$ with $x \leq x'$. For all $n$, $f(x - r_n) = 1$ implies $r_n \leq x$, so $r_n \leq x'$ and $f(x' - r_n) = 1 = f(x - r_n)$. Since $f$ is either $0$ or $1$, $f(x - r_n)/n^2 \leq f(x' - r_n)/n^2$. Therefore $F(x) \leq F(x')$ since all terms of the series for $F(x)$ are bounded by those of $F(x')$.

      \paragraph{Discontinuity} We will show that $F$ is discontinous at all $r_n$. Choose an arbitrary $\delta > 0$. Set $x = \max\{0, r_n - \delta/2\}$, so $\abs{r_n - x} \leq \delta/2 < \delta$\footnote{We never have $x = r_n$ because $\delta > 0$ and $r_n > 0$.}. But $\abs{F(r_n) - F(x)} \geq 1/n^2$. This can be seen by noting that for all $m$, $f(x - r_m) \leq f(r_n - r_m)$ since $x < r_n$, and in particular, $0 = f(x - r_n) < f(r_n - r_n) = 1$. Then the series for $F(r_n)$ includes all the terms that the series for $F(x)$ includes, but $F(r_n)$ also has a $1/n^2$ term that $F(x)$ is missing. \qed

    \item Choose a countable dense sequence $\{r_n\}$ in $[0,1]$. Let\[
        F(x) = \sum_{n=1}^\infty 3^{-n}g(x-r_n)
      \] with \[
        g(x) = \begin{cases}\
          \sin 1/x & \text{if }x \neq 0\text{,} \\
          0 & \text{if }x = 0\text{.}
        \end{cases}
      \] We want to show that $F$ is integrable, discontinuous at every point of $\{r_n\}$, and non-monotonic in any subinterval of $[0,1]$.

      For convenience, define $g_n : [0, 1] \to \bbR$ by $g_n(x) = g(x - r_n)$.

      \paragraph{Integrability} It suffices to show that the set of discontinuities of $F$ has measure zero.

      Let $S = [0,1]\setminus \{r_n\}_{n=1}^\infty$. We will show that $F$ is continuous at every $x \in S$. Choose an abitrary $x \in S$ and $\epsilon > 0$.

      Let $N$ be a positive integer such that $3^{-N} < \epsilon/2$. For each $n = 1, \dots, N$, $g_n(x)$ is continuous everywhere except at $x = r_n$. Since $x \neq r_n$ by our choice of $S$, there exists $\delta_n$ such that $\abs{g_n(x) - g_n(x')} < \epsilon$ whenever $\abs{x - x'} < \delta_n$.

      Set $\delta = \min\{\delta_1, \dots, \delta_N\}$. Then for all $x' \in [0, 1]$, $\abs{x - x'} < \delta$ implies \begin{align*}
        \abs{F(x) - F(x')}
        &\leq \sum_{n=1}^\infty 3^{-n} \abs{g_n(x) - g_n(x')} \\
        &= \sum_{n=1}^N 3^{-n} \abs{g_n(x) - g_n(x')} \\
        &\qquad+ \sum_{n=N+1}^\infty 3^{-n} \abs{g_n(x) - g_n(x')} \text{.}
        % &\leq \epsilon\sum_{n=1}^N 3^{-n} + 2\sum_{n=N+1}^\infty 3^{-n} \\
        % &= \epsilon\frac{1 - 3^{-N}}{2} + 3^{-N} \\
        % &< \frac12\epsilon + 2\frac{\epsilon}{4} = \epsilon\text{.}
      \end{align*} The first term is bounded by \[
        \sum_{n=1}^N 3^{-n} \abs{g_n(x) - g_n(x')}
        < \epsilon\sum_{n=1}^N 3^{-n}
        = \epsilon \frac{1 - 3^{-N}}{2} < \frac{\epsilon}{2}
      \] by our choice of $\delta$. The second term is bounded by \[
        \sum_{n=N+1}^\infty 3^{-n} \abs{g_n(x) - g_n(x')}
        \leq 2 \sum_{n=N+1}^\infty 3^{-n} = 3^{-N} < \frac{\epsilon}{2}
      \] because $g_n$ is bounded by $1$ and by our choice of $N$. Hence, \[
        \abs{F(x) - F(x')} < \frac{\epsilon}{2} + \frac{\epsilon}{2} = \epsilon \text{.}
      \]

      Then the set of discontinuities is a subset of the countable set $[0,1]\setminus S = \{r_n\}_{n=1}^\infty$, and therefore has measure zero.

      \paragraph{Discontinuity} Choose some positive integer $n$. We will show that $F$ is not continuous at $r_n$. Choose an arbitary $\delta > 0$.

      Let \[
        h(x) = \sum_{k=1}^{n-1} 3^{-k} g_k(x) + 3^{-n-1}g_{n+1}(x) \text{.}
      \] Our choice of $h$ is continuous at $r_n$ as a finite sum of functions that are continuous at $r_n$. Hence, there exists $\delta_h>0$ such that $\abs{x - r_n} < \delta_h$ implies $\abs{h(x) - h(r_n)}<3^{-n-1}$. Refine our choice of $\delta$ as $\delta' = \min\{\delta, \delta_h\}$.

      We can choose an integer $m$ such that $-\delta' < 1/\pi(m + 1/2) < \delta'$ and $0 \leq r_n + 1/\pi(m + 1/2) \leq 1$. Let $x = r_n + 1/\pi(m + 1/2)$ and observe that $\abs{r_n - x} < \delta' \leq \delta'$.

      Then by the reverse triangle inequality,\begin{align*}
        \abs{F(r_n) - F(x)}
        &\geq3^{-n}\abs{g_n(r_n) - g_n(x)} \\
        &\qquad- \abs{h(r_n) - h(x)} \\
        &\qquad- \sum_{k=n+2}^\infty 3^{-k}\abs{g_k(r_n) - g_k(x)} \text{.}
      \end{align*} We simplify the first term with\begin{align*}
        3^{-n}\abs{g_n(r_n) - g_n(x)}
        &= 3^{-n}\abs{g(0) - g\left(\frac{1}{m(\pi + 1/2)}\right)} \\
        &= 3^{-n}\abs{0 - \sin\pi \left(m + \frac12\right)} \\
        &= 3^{-n} \text{.}
      \end{align*} By our choice of $\delta_h$ and thus $\delta'$, $\abs{h(r_n) - h(x)} < 3^{-n-1}$. Finally, $\abs{g_k}$ has a bound at $1$, so \[
        \sum_{k=n+2}^\infty 3^{-k}\abs{g_k(r_n) - g_k(x)}
        \leq 2\sum_{k=n+2}^\infty 3^{-k}
        = 3^{-n-1} \text{.}
      \] Combining these bounds, \[
        \abs{F(r_n) - F(x)} > 3^{-n} - 3^{-n-1} - 3^{-n-1} = 3^{-n-1} \text{,}
      \] which is a constant not dependent on $\delta$.

      \paragraph{Non-monotonicity} Choose arbitrary $0 \leq a < b \leq 1$. We will find $a < y < y' < b$ so that $F(y) < F(y')$ and also $a < z < z' < b$ with $F(z) > F(z')$.

      Since $\{r_k\}_{k=1}^\infty$ is dense in $[0,1]$, $(a, b) \cap \{r_k\}_{k=1}^\infty$ is nonempty. Let $n$ be the index of the first element of 
      $\{r_k\}_{k=1}^\infty$ contained in $(a, b)$; that is, $r_n \in (a, b)$ but $r_k \notin (a, b)$ for $k= 1, \dots, n-1$.

      Set $y = z = r_n$. Define $h : [0, 1] \to \bbR$ by \[
        h(x) = \sum_{k=1}^{n-1} 3^{-k} g_k(x) + 3^{-n-1} g_{n+1}(x) \text{.}
      \] We have that $h$ is continuous at $r_n$ as a sum of continuous functions. Then there exists $\delta > 0$ such that $\abs{x - r_n} < \delta$ implies $\abs{h(x) - h(r_n)} < 3^{-n-1}$ and $\delta$ is small enough such that $(r_n - \delta, r_n + \delta) \subseteq (a, b)$. Let $y' = r_n + 1/(2k_y + 1/2)\pi$, where $k_y > 0$ is an integer such that $1/(2k_y + 1/2)\pi < \delta$. Similarly, let $z' = r_n + 1/(2k_z + 3/2)\pi$, for integer $k_z > 0$ such that $1/(2k_z + 3/2)\pi < \delta$.

      Then \begin{align*}
        F(y) - F(y')
        &= 3^{-n}g_n(r_n) - 3^{-n}g_n\left(r_n + \frac1{\left(2k_y + \frac12\right)\pi}\right) \\
        &\qquad + h(y) - h(y') \\
        &\qquad + \sum_{k=n+2}^\infty 3^{-k} g_k(y) - \sum_{k=n+2}^\infty 3^{-k} g_k(y') \text{.}
      \end{align*} The first term can be simplified as\begin{align*}
        &3^{-n}g_n(r_n) - 3^{-n}g_n\left(r_n + \frac1{\left(2k_y + \frac12\right)\pi}\right) \\
        &= 3^{-n}\sin0 - 3^{-n}\sin\left(2k_y + \frac12\right)\pi \\
        &= -3^{-n}
      \end{align*} The second term can be bounded with $\abs{h(y) - h(y')} < 3^{-n-1}$ by our choice of $\delta$ and hence $y'$. We bound the last term as \begin{align*}
        &\abs{\sum_{k=n+2}^\infty 3^{-k} g_k(y) - \sum_{k=n+2}^\infty 3^{-k} g_k(y')} \\
        &\leq \sum_{k=n+2}^\infty 3^{-k} \abs{g_k(y)}  \sum_{k=n+2}^\infty 3^{-k} \abs{g_k(y')} \\
        &\leq 2\sum_{k=n+2}^\infty 3^{-k} \\
        &= 3^{-n-1}
      \end{align*} Lastly, \begin{align*}
        F(y) - F(y')
        &\leq 3^{-n}g_n(r_n) - 3^{-n}g_n\left(r_n + \frac1{\left(2k_y + \frac12\right)\pi}\right) \\
        &\qquad + \abs{h(y) - h(y')} \\
        &\qquad + \abs{\sum_{k=n+2}^\infty 3^{-k} g_k(y) - \sum_{k=n+2}^\infty 3^{-k} g_k(y')} \\
        &< -3^{-n} + 3^{-n-1} + 3^{-n-1} = -3^{-n-1} \text{,}
      \end{align*} so $F(y) < F(y')$.

      Similarly, for $z$, the first term simplifies to\[
        3^{-n}g_n(r_n) - 3^{-n}g_n\left(r_n + \frac1{\left(2k_z + \frac32\right)\pi}\right)
        = 3^{-n} \text{,}
      \] the second term has a bound of $\abs{h(z) - h(z')} < 3^{-n-1}$, and the last term has can be bounded with \[
        \abs{\sum_{k=n+2}^\infty 3^{-k} g_k(z) - \sum_{k=n+2}^\infty 3^{-k} g_k(z')} = 3^{-n-1} \text{.}
      \] These combine to\begin{align*}
        F(z) - F(z')
        &\geq 3^{-n}g_n(r_n) - 3^{-n}g_n\left(r_n + \frac1{\left(2k_z + \frac12\right)\pi}\right) \\
        &\qquad - \abs{h(z) - h(z')} \\
        &\qquad - \abs{\sum_{k=n+2}^\infty 3^{-k} g_k(z) - \sum_{k=n+2}^\infty 3^{-k} g_k(z')} \\
        &> 3^{-n} - 3^{-n-1} - 3^{-n-1} = 3^{-n-1} \text{,}
      \end{align*} so $F(z) > F(z')$. \qed

    \item We show that $F$ is integrable on every interval in $\bbR$ but is discontinuous whenever $x = m/2n$ for odd $m \in \bbZ$ and nonzero $n \in \bbZ$.

    \paragraph{Integrability} Since $F$ is periodic, it suffices to show that it is integrable on $[-1/2, 1/2]$. Define a sequence of partitions $P_1, P_2, \dots$ by \[
      P_n = \left\{-1/2, -1/2 + \frac{1}{n!}, \dots, 1/2 - \frac{1}{n!}, 1/2\right\} \subset [-1/2, 1/2] \text{.}
    \]

    Let $F_n(x) = \sum_{k=1}^n (kx)/k^2$ denote the partial sums. Then \begin{align*}
      \calU(P_n, F) - \calL(P_n, F)
      &\leq \left[\calU(P_n, F_n) - \calL(P_n, F_n)\right] \\
      &\qquad+ \left[\calU(P_n, F - F_n) - \calL(P_n, F - F_n)\right] \text{.}
    \end{align*}

    We can easily bound \[
      \calU(P_n, F - F_n) - \calL(P_n, F - F_n)
      \leq 2\sum_{k=n+1}^\infty \abs{\frac{(kx)}{k^2}}
      \leq \sum_{k=n+1}^\infty \frac{1}{k^2}
      \text{,}
    \] so $\calU(P_n, F - F_n) - \calL(P_n, F - F_n) \to 0$ as $n \to \infty$.

    To bound $\calU(P_n, F_n) - \calL(P_n, F_n)$, observe that each $x \mapsto (kx)/k^2$ is piecewise affine with breaks at $\{-1/2, -1/2 + 1/k, \dots, 1/2-1/k, 1/2\} \subseteq P_n$ when $n \geq k$. Thus $F_n$ is affine within the interior of each interval in the partitioning by $P_n$, with a gradient of $\sum_{k=1}^n k/k^2 = H_n$, the $n$\textsuperscript{th} harmonic number. Each interval has width $1/n!$, yielding a height of $H_n/n!$ and an area of $H_n/(n!)^2$. Finally, since there are $n!$ intervals in our partitioning, $\calU(P_n, F_n) - \calL(P_n, F_n) = H_n/n!$, which tends to $0$ as $n \to \infty$.\footnote{If we insist on following exactly the definition of $\calL$ and $\calU$ in the appendix, then we need to formally account for the discontinuities at interval endpoints. There is one jump of length $1$, two of length $1/4$, three of length $1/9$, and so on, potentially overlapping. A jump of length $\ell$ may increase the height of a bounding box by at most $\ell$. Since each bounding box has width $1/n!$, this pedantic threatment of jump points increases $\calU(P_n, F_n) - \calL(P_n, F_n)$ by at most $(\sum_{k=1}^n k/k^2)/n! = H_n/n!$, which tends to $0$ as $n\to \infty$.}

    Hence, $\calU(P_n, F) - \calL(P_n, F) \to 0$ as $n\to\infty$ and $F$ is integrable.

    \paragraph{Discontinuity} Let $x = m/2n$ for $m, n \in \bbZ$ with $m$ odd and $n \neq 0$. We show that the right limit $\lim_{t\to x^+}F(t)$ exists and does not equal $F(x)$.

    We can bound \[
      F(t) = \sum_{k=1}^\infty \frac{(kt)}{k^2} \leq \sum_{k=1}^\infty \frac{1}{k^2}, \text{,}
    \] so the series converges uniformly in $t$. We may thus swap the limits and \[
      \lim_{t\to x^+}F(t) = \sum_{k=1}^\infty \lim_{t\to x^+} \frac{(kt)}{k^2} \text{.}
    \] For each $k$, $\lim_{t\to x^+} (kt)/k^2$ equals either $(xt)/k^2$ or $1/2k^2$. If every term equals the former, then $\lim_{t\to x^+}F(t) = F(x)$, and if $\lim_{t\to x^+} (kt)/k^2 = 1/2k^2$ for at least one $k$, then $F(x) > \lim_{t\to x^+}F(t)$.

    Consider $k = n$. We can write $1/2 - m/2n = (n - m)/2n$. Since $n$ and $m$ are odd $n - m$ is even, and $1/2 - x$ is an integer multiple of $1/n$. Thus $\lim_{t\to x^+} (nt)/n^2 = 1/2n^2$ and $F(x) > \lim_{t\to x^+}F(t)$, so $F$ is discontinuous at $x$. \qed
  \end{enumerate}
\end{document}
