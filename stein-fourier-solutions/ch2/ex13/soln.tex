\documentclass[oneside]{article}

\usepackage{amsfonts}
\usepackage{amsmath}
\usepackage{amssymb}
\usepackage{amsthm}
\usepackage{enumitem}
\usepackage{mathtools}

\allowdisplaybreaks

% Notation shortcuts
\newcommand\abs[1]{\left|#1\right|}
\newcommand\defeq{\overset{\mathrm{def}}{=}}
\newcommand*\Laplace{\mathop{}\!\mathbin\bigtriangleup}

\DeclarePairedDelimiter\ceil{\lceil}{\rceil}
\DeclarePairedDelimiter\floor{\lfloor}{\rfloor}

\DeclareMathOperator{\sgn}{sgn}
\DeclareMathOperator{\atantwo}{atan2}

\newcommand\bbC{\mathbb{C}}
\newcommand\bbR{\mathbb{R}}
\newcommand\bbZ{\mathbb{Z}}

\newtheorem*{lem}{Lemma}
\newtheorem*{cor}{Corollary}

\renewcommand{\thefootnote}{[\arabic{footnote}]}

\begin{document}
  \begin{enumerate}[label=(\alph*)]
    \item We will prove this in three parts. For parts 1 and 2 we will assume
    that $s = 0$. In part 1, we will show that \[
      \sum_{n=1}^\infty c_nr^n = (1-r) \sum_{n=1}^\infty s_nr^n \text{.}
    \] In part 2, we prove that \[
      \lim_{r\to1}(1-r) \sum_{n=1}^\infty s_nr^n = 0 \text{.}
    \] Finally, in part 3, we generalise this result to cases where $s \neq 0$.

    \begin{enumerate}[label=\textbf{\arabic*.}]
      \item
        Let $s_n$ be the partial sum $s_n = \sum_{k=1}^n c_k$.

        We have \begin{align*}
          (1-r)\sum_{n=1}^Ns_nr^n % + s_Nr^{N+1}
          &= \sum_{n=1}^Ns_nr^n - \sum_{n=1}^Ns_nr^{n+1} \\
          &= \sum_{n=1}^Ns_nr^n - \sum_{n=2}^{N+1}s_{n-1}r^n \\
          &= s_1r + \sum_{n=2}^N(s_n - s_{n-1})r^n - s_Nr^{N+1} \\
          &= c_1r + \sum_{n=2}^Nc_nr^n - s_Nr^{N+1} \\
          &= \sum_{n=1}^Nc_nr^n - s_Nr^{N+1} \text{.}
        \end{align*} Hence \[
          \sum_{n=1}^Nc_nr^n = (1-r)\sum_{n=1}^Ns_nr^n + s_Nr^{N+1} \text{.}
        \]

        By assumption $s_N \to 0$ as $N\to\infty$. Also, $r^{N+1} \to 0$ as
        $N\to\infty$ because $r \in (0, 1)$. Hence, $s_Nr^{N+1}$ vanishes as
        $N \to \infty$.

        For all $0 < r < 1$, L.H.S. converges by Dirichlet's test since $s_n$ is
        bounded and $r^n$ converges monotonically to $0$. Then $R.H.S.$ also
        converges and we get\[
          \sum_{n=1}^\infty c_nr^n = (1-r)\sum_{n=1}^\infty s_nr^n \text{.}
        \]

      \item We show that $\lim_{r\to0}\sum_{n=1}^\infty s_nr^n=0$. We argue by
        the Moore-Osgood theorem that the limit can be interchanged with the
        sum.

        Firstly we observe that for all $N$ finite,
        $\lim_{r\to1}(1-r)\sum_{n=1}^N s_nr^n$ exists.

        Secondly, we need to show that $\sum_{n=1}^\infty s_nr^n$ converges
        uniformly for all $r\in(0,1)$. We have shown convergence in part 1. To
        prove that it is uniform, choose some $\epsilon > 0$. Let $N$ be such
        that $\abs{s_n} < \epsilon$ for all $n > N$. Then for all $r \in (0,1)$,
        \begin{align*}
          \abs{(1-r)\sum_{n=1}^N s_nr^n - (1-r)\sum_{n=1}^\infty s_nr^n}
          &= \abs{(1-r)\sum_{n=N+1}^\infty s_nr^n} \\
          &\leq (1-r)\epsilon\sum_{n=N+1}^\infty r^n \\
          &= (1-r)\epsilon\frac{r^{N+1}}{1-r} \\
          &< \epsilon\text{,}
        \end{align*} so the sum converges uniformly.

        Hence, we can interchange the limits by the Moore-Osgood theorem and
        \begin{align*}
          \lim_{r\to1}\sum_{n=1}^\infty c_nr^n
          &= \lim_{r\to1}(1-r)\sum_{n=1}^\infty s_nr^n \\
          &= \sum_{n=1}^\infty s_n\lim_{r\to1}\left(r^n - r^{n+1}\right) \\
          &= 0 \text{.}
        \end{align*} Therefore, $\sum c_n$ is Abel summable to $0$.

      \item We have shown that if a series converges to $0$, then it is also
        Abel summable to $0$. We now generalise this to series that converge to
        $s \neq 0$.

        Let $\sum_{n=1}^\infty c_n = s$, then construct $\{c_n'\}_{n=1}^\infty$
        by $c'_1 = -s$, $c'_2 = c_1$, $c'_3 = c_2$, and so on. In other words
        we are constructing a new series by prepending $-s$. Then \[
          \sum_{n=1}^\infty c'_n
          = -s + \sum_{n=2}^\infty c'_n
          = -s + \sum_{n=1}^\infty c_n
          = 0 \text{.}
        \] Since $\sum c_n'$ converges to $0$ it is also Abel summable to $0$
        and \[
          0 = \lim_{r\to1}\sum_{n=1}^\infty c_n'r^n
          = \lim_{r\to1}c_1'r + \lim_{r\to1}\sum_{n=2}^\infty c_n'r^n
          = -s + \lim_{r\to1}\sum_{n=1}^\infty c_nr^n \text{,}
        \] so $\sum c_n$ is Abel summable to $s$.\qed
    \end{enumerate}

  \item Consider the series $\sum_{n=0}^\infty (-1)^n$. This series clearly
    diverges because $\lim_{n\to\infty}(-1)^n$ does not converge to $0$.
    However, this series is Abel summable. For all $r \in [0, 1)$ define,\[
        A(r)
        = \sum_{n=0}^\infty (-1)^nr^n
        = \sum_{n=0}^\infty (-r)^n
        = \frac{1}{1+r}
      \] where for the last inequality we recognise this as a geometric series.
      Then $\lim_{r\to1}A(r) = 1/2$.

  \item The proof is similar to (a). We first assume that $\sigma = 0$. In part
    1 we show that  \[
      \sum_{n=1}^\infty c_nr^n = (1-r)^2\sum_{n=1}^\infty n \sigma_n r^n\text{.}
    \] Part 2 we prove that \[
      \lim_{r\to1}(1-r)^2\sum_{n=1}^\infty n \sigma_n r^n = 0\text{.}
    \] Finally, part 3 generalises this result to cases where $\sigma \neq 0$.

    \begin{enumerate}[label=\textbf{\arabic*.}]
      \item Let $s_n$ be the partial sum $s_n = \sum_{k=1}^n c_k$. Let
        $\sigma_n = \frac1n\sum_{k=1}^ns_k$ be the mean of fit $n$ partial sums.

        Following the same argument as in (a), part 1, for $r\in(0,1)$ we have
        \begin{align*}
          &\sum_{n=1}^Nc^nr^n = (1-r)\sum_{n=1}^N s_nr^n + s_Nr^{N+1}
          \text{,} \\
          &\sum_{n=1}^Ns^nr^n
          = (1-r)\sum_{n=1}^N n\sigma_nr^n + N\sigma_Nr^{N+1}
          \text{.}
        \end{align*}

        Then\begin{align*}
          \sum_{n=1}^\infty c_nr^n
          &= (1-r)^2 \sum_{n=1}^\infty n\sigma_nr^n \\
          &\quad + (1-r)\lim_{n\to\infty}s_nr^{n+1}
          + \lim_{n\to\infty}n\sigma_nr^{n+1}
          \text{.}
        \end{align*}

        By assumption, $\sigma_n\to0$ as $n\to\infty$. Also, $nr^{n+1}\to0$
        since $r \in (0,1)$. Hence, \begin{equation}
          \label{eq:last-term-vanishes}
          \lim_{n\to\infty}n\sigma_nr^{n+1}=0\text{,}
        \end{equation}and the last
        term vanishes.

        To show that the middle term vanishes as well, observe that
        $s_n = n\sigma_n - (n - 1)\sigma_{n-1}$. Then \[
          \lim_{n\to\infty}s_nr^{n+1}
          = \lim_{n\to\infty}n\sigma_nr^{n+1}
          - \lim_{n\to\infty}(n - 1)\sigma_{n-1}r^{n+1}
          = 0\text{,}
        \] where the last equality is by \eqref{eq:last-term-vanishes}.

        Hence,\[
          \sum_{n=1}^\infty c_nr^n = (1-r)^2 \sum_{n=1}^\infty n\sigma_nr^n
          \text{.}
        \]

        Finally, we show that these series converge for all $r \in (0,1)$. For
        $n$ big enough $\abs{\sigma_n} < 1$, so
        $\abs{n\sigma_nr^n} < \abs{nr^n}$. Then R.H.S. converges because
        $\sum \abs{nr^n}$ converges.

      \item We show that
        $\lim_{r\to1} (1-r)^2\sum_{n=1}^\infty n \sigma_nr^n = 0$.

        From part 1, we know that $(1-r)^2 \sum_{n=1}^\infty n\sigma_n r^n$
        converges for all $r \in (0, 1)$. To show
        that this convergence is uniform in $r$, pick an $\epsilon > 0$. Let
        $N$ be such that $\abs{\sigma_n} < \epsilon$ for all $n > N$. Then for
        all $r \in (0, 1)$,\begin{align*}
          \abs{(1-r)^2 \sum_{n=1}^\infty N\sigma_nr^n
          - (1-r)^2 \sum_{n=1}^\infty n\sigma_nr^n}
          &= \abs{(1-r)^2 \sum_{n=N+1}^\infty n\sigma_nr^n} \\
          &\leq (1-r)^2 \sum_{n=N+1}^\infty nr^n\abs{\sigma_n} \\
          &\leq (1-r)^2 \epsilon\sum_{n=N+1}^\infty nr^n \\
          &\leq (1-r)^2 \epsilon\sum_{n=1}^\infty nr^n \\
          &= (1-r)^2 \epsilon\frac{r}{(1-r)^2} \\
          &\leq \epsilon \text{.}
        \end{align*} Above, we use the fact that for $\abs{r} < 1$,
        \[
          \sum_{n=1}^M nr^{n-1}
          = \frac{d}{dr}\sum_{n=1}^M r^n
          = \frac{d}{dr}\frac{r-r^{M+1}}{1-r}
          = \frac{Mr^{M+1}-(M+1)r^M+1}{(1-r)^2} \text{.}
        \] Limiting $M \to \infty$, we obtain\begin{equation}
          \label{eq:infinite-sum-nr}
          \sum_{n=1}^\infty nr^{n-1}
          = \frac{1}{(1-r)^2} \text{.}
        \end{equation}

        Then the sum converges uniformly and we can interchange the limits by
        the Moore-Osgood theorem. We have \begin{align*}
          \lim_{r\to1}\sum_{n=1}^\infty c_nr^n
          &= \lim_{r\to1}(1-r)^2\sum_{n=1}^\infty n\sigma_nr^n \\
          &= \sum_{n=1}^\infty \lim_{r\to1}n\sigma_nr^n(1-r)^2 \\
          &= 0
        \end{align*} as desired.

        \item We have shown that if $\sum c_n$ is Cesàro summable to $0$, then
        it is also Abel summable to $0$. We now generalise this to series that
        are Cesàro summable to $\sigma \neq 0$.

        If $\sum c_n$ is Cesàro summable to $\sigma$, then construct
        $\{c_n'\}_{n=1}^\infty$ by $c'_1 = c_1-s$, $c'_2 = c_2$, $c'_3 = c_3$,
        and so on. In other words we are constructing a new series by
        subtracting $s$ from the first term and leaving the others intact. Then
        the partial sums $\sigma'_n$ satisfy $\sigma'_n = \sigma_n - \sigma$, so
        $\sigma_n' \to 0$ as $n \to \infty$. Since $\sum c_n'$ is Cesàro
        summable to $0$, it is also Abel summable to $0$ and \[
          0 = \lim_{r\to1}\sum_{n=1}^\infty c_n'r^n
          = \lim_{r\to1}c_1'r + \lim_{r\to1}\sum_{n=2}^\infty c_n'r^n
          = c_1 -s + \lim_{r\to1}\sum_{n=2}^\infty c_nr^n \text{,}
        \] so $\sum c_n$ is Abel summable to $s$.\qed
    \end{enumerate}

  \item Consider the sum \[
    \sum_{n=1}^\infty (-1)^{n-1}n \text{.}
  \] We show that it is Abel summable but not Cesàro summable.

  To show Abel summability, we use \eqref{eq:infinite-sum-nr} to find that\[
    \sum_{n=1}^\infty (-1)^{n-1}nr^n = \frac{r}{(1+r)^n} \text{.}
  \] Then\[
    \lim_{r\to1}\sum_{n=1}^\infty (-1)^{n-1}nr^n
    = \lim_{r\to1}\frac{r}{(1+r)^n}
    = \frac{1}{4} \text{,}
  \] so $\sum_{n=1}^\infty (-1)^{n-1}n$ is Abel summable to $1/4$.

  To show that it is not Cesàro summable, observe that the partial sums are
  \[
    \sigma_n = \begin{cases}
      (n + 1) / 2 & \text{ for $n$ odd,} \\
      -n / 2 & \text{ for $n$ even.}
    \end{cases}
  \] This is easily shown by induction. For $n$ odd, we have
  $\sigma_n = \sigma_{n-1} + n = -(n - 1) / 2 + n = (n+1)/2$. For $n$ even, we
  get $\sigma_n = \sigma_{n-1} - n = n/2 - n = -n / 2$. $\sigma_1 = 1$
  forms the base case. Then $\sigma_n$ diverges and
  $\sum_{n=1}^\infty (-1)^{n-1}n$ is not Cesàro summable.

  \end{enumerate}
\end{document}
