\documentclass[oneside]{article}

\usepackage{amsfonts}
\usepackage{amsmath}
\usepackage{amssymb}
\usepackage{amsthm}
\usepackage{enumitem}
\usepackage{mathtools}

\allowdisplaybreaks

% Notation shortcuts
\newcommand\abs[1]{\left|#1\right|}
\newcommand\defeq{\overset{\mathrm{def}}{=}}
\newcommand*\Laplace{\mathop{}\!\mathbin\bigtriangleup}

\DeclareMathOperator{\sgn}{sgn}
\DeclareMathOperator{\atantwo}{atan2}

\newcommand\bbC{\mathbb{C}}
\newcommand\bbR{\mathbb{R}}
\newcommand\bbZ{\mathbb{Z}}

\newtheorem*{lem}{Lemma}

\begin{document}
Let \[
  A_m = \frac{2h}{m^2}\frac{\sin mp}{p(\pi - p)}
  \qquad\text{and}\qquad
  f(x) = \begin{cases}
    \frac{xh}{p} &\text{for } 0 \leq x \leq p\text{,} \\
    \frac{h(\pi-x)}{\pi-p} &\text{for } p \leq x \leq \pi \text{.}
  \end{cases}
\]

Let $C = 2h/p(\pi-p) \geq 0$. Then \[
  \abs{A_m} = C\frac{\abs{\sin mp}}{m^2} \leq \frac{C}{m^2} \text{,}
\] so \[
  \abs{A_m \sin mp} = \abs{A_m}\abs{\sin mx} \leq \frac{C}{m^2} \text{.}
\] Since $\sum_{m=1}^\infty 1/m^2$ converges, \[
  \sum_{m=1}^\infty A_m \sin mp
\] converges absolutely.

We also know that $f$ is continuous and by chapter 1, exercise 9, we have that
$\sum_{m=1}^\infty A_m \sin mp$ is the Fourier series of $f$. Then by corollary
2.3,\[
  f(x) = \sum_{m=1}^\infty A_m \sin mp
\] for all $x \in [0, \pi]$.\qed

\end{document}
