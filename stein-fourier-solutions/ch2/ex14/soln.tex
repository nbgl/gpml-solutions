\documentclass[oneside]{article}

\usepackage{amsfonts}
\usepackage{amsmath}
\usepackage{amssymb}
\usepackage{amsthm}
\usepackage{enumitem}
\usepackage{mathtools}
\usepackage{cleveref}

\allowdisplaybreaks

% Notation shortcuts
\newcommand\abs[1]{\left|#1\right|}
\newcommand\defeq{\overset{\mathrm{def}}{=}}
\newcommand*\Laplace{\mathop{}\!\mathbin\bigtriangleup}

\DeclarePairedDelimiter\ceil{\lceil}{\rceil}
\DeclarePairedDelimiter\floor{\lfloor}{\rfloor}

\DeclareMathOperator{\sgn}{sgn}
\DeclareMathOperator{\atantwo}{atan2}

\newcommand\bbC{\mathbb{C}}
\newcommand\bbR{\mathbb{R}}
\newcommand\bbZ{\mathbb{Z}}

\newtheorem*{lem}{Lemma}
\newtheorem*{cor}{Corollary}

\renewcommand{\thefootnote}{[\arabic{footnote}]}

\begin{document}
  \begin{enumerate}[label=(\alph*)]
    \item We show that $s_n - \sigma_n \to 0$ as $n \to \infty$.

    We require the fact that \[
      \sum_{k=1}^n s_k
      = \sum_{k=1}^n \sum_{j=1}^k c_j
      = \sum_{k=1}^n (n - k + 1)c_k
      \text{,}
    \] so \begin{equation}
      \label{eq:diff}
      s_n - \sigma_n
      = \sum_{k=1}^n c_k - \frac1n\sum_{k=1}^n (n - k + 1)c_k
      = \frac1n\sum_{k=1}^n (k - 1)c_k \text{.}
    \end{equation}

    Pick an arbitrary $\epsilon > 0$. We will show that there exists $N$ such
    that for all $n > N$, $\abs{s_n - \sigma_n} < \epsilon$.
    
    Since $nc_n \to 0$ by assumption, there exists $M$ such that
    $k\abs{c_k} < \epsilon/2$ for all $k > M$. Let
    $\rho = \sum_{k=1}^M k \abs{c_k}$.

    Choose $N \geq \max(2\rho/\epsilon, M)$. Then for all $n > N$ we have
    $n > 2\rho/\epsilon$, so $n\epsilon > \rho+n\epsilon/2$. Hence,
    \begin{align*}
      n\abs{s_n-\sigma_n}
      &= \abs{\sum_{k=1}^n (k-1)c_k} & \text{(by \eqref{eq:diff})} \\
      &\leq \sum_{k=1}^n (k-1)\abs{c_k} \\
      &\leq \sum_{k=1}^n k\abs{c_k} \\
      &= \rho + \sum_{k=M+1}^n k\abs{c_k} \\
      &\leq \rho + \sum_{k=M+1}^n \frac{\epsilon}{2} \\
      &= \rho + (n-M)\frac{\epsilon}{2} \\
      &\leq \rho + n\frac{\epsilon}{2} \\
      &< n\epsilon \text{.}
    \end{align*} Dividing both sides by $n$ concludes the proof.\qed

  \item
    We are given that $\lim_{k\to\infty}k\abs{c_k} = 0$ and that $\{c_k\}_{k=1}^\infty$ is Abel summable.

    Define \begin{align*}
      A_n(r) &= \sum_{k=1}^n c_nr^k \text{,} \\
      A(r) &= \lim_{n\to\infty}A_n(r) = \sum_{k=1}^n c_nr^k \text{,} \\
      \overline{A} &= \lim_{r\to1} A(r) \text{.}
    \end{align*}

    We want to show that $A(1) = \sum_{k=1}^\infty c_k$ converges. 

    Choose an arbitrary $\epsilon > 0$. We will show that there exists $N$ such that for all $n > N$, $\abs{A_n(1) - \overline{A}} < \epsilon$.

    Let $\rho > 0$ such that $k\abs{c_k} < \rho$ for all $k = 1,2,\dots$; this bound exists because $\lim_{k\to\infty}k\abs{c_k} = 0$. Since $\lim_{t\to1} A(t) = \overline{A}$, there exists $0 < T < 1$ such that for all $T < t < 1$, $\abs{A(t) - \overline{A}} < \epsilon/3$. Let $\alpha > \max\{3\rho/\epsilon, 1/(1-T)\} > 0$. Because $\lim_{k\to\infty} k\abs{c_k}=0$, we can find $N$ such that for all $k>N$, $k\abs{c_k} < \epsilon/3\alpha$.

    Choose an arbitrary $n > N$. We will show that $\abs{A_n(1) - \overline{A}} < \epsilon$.

    Let $r = 1 - 1/\alpha n$. By the triangle inequality\[
      \abs{A_n(1) - \overline{A}}
      \leq \abs{A_n(1) - A_n(r)}  + \abs{A_n(r) - A(r)} + \abs{A(r) - \overline{A}}
      \text{.}
    \] We tackle these terms one by one.

    Firstly,\begin{align}
      \abs{A_n(1) - A_n(r)}
      &= \abs{\sum_{k=1}^n c_k r^k - \sum_{k=1}^n c_k} \nonumber\\
      &= \abs{\sum_{k=1}^n c_k (1 - r^k)} \nonumber\\
      &= (1-r)\abs{\sum_{k=1}^n c_k \sum_{j=0}^{k-1} r^j} \label{eq:factor}\\
      &\leq (1-r)\sum_{k=1}^n k\abs{c_k} \label{eq:rlessthanone} \\
      &< (1-r)n\rho \label{eq:bounded}\\
      &= \frac1{\alpha n}n\rho = \frac{\rho}{\alpha} < \frac{\epsilon}{3} \text{.} \label{eq:sub}
    \end{align} In \cref{eq:factor} we use the fact that $1 - r^k = (1-r)(1 + r + \dots + r^{k-1})$. In \cref{eq:rlessthanone} $0 < r^j < 1$ implies $\sum_{j=0}^{k-1} r^j \leq k$. \Cref{eq:bounded} is by the bound that defines $\rho$. Finally in \cref{eq:sub} we substitute for $r$ and $\alpha$.

    Bounding the second term, \begin{align}
      \abs{A(r) - A_n(r)}
      &= \abs{\sum_{k=n+1}^\infty c_k r^k} \nonumber\\
      &< \frac{\epsilon}{3\alpha} \sum_{k=n+1}^\infty \frac1k r^k \label{eq:bounded2}\\
      &< \frac{\epsilon}{3\alpha n}\sum_{k=n+1}^\infty r^k \label{eq:nlessthank}\\
      &= \frac{\epsilon r^{n+1}}{3\alpha n (1-r)} \nonumber\\
      &< \frac{\epsilon}{3\alpha n (1-r)}  \nonumber\\
      &= \frac{\epsilon}{3} \text{.} \label{eq:sub2}
    \end{align} In \cref{eq:bounded2} we use the fact that $k > n > N$, so by our choice of $N$, $k\abs{c_k} < \epsilon/3\alpha$. \Cref{eq:nlessthank} is because $1/k < 1/n$. In \cref{eq:sub2} we substitute for $r$.

    For the final term, we have $\alpha > 1/(1-T)$, so $r = 1 - 1/\alpha n > T$. Then $\abs{A(r) - \overline{A}} < \epsilon/3$ by our choice of $T$.

    Putting it all together, \begin{align*}
      \abs{A_n(1) - \overline{A}}
      &\leq \abs{A_n(1) - A_n(r)}  + \abs{A_n(r) - A(r)} + \abs{A(r) - \overline{A}} \\
      &< \epsilon/3 + \epsilon/3 + \epsilon/3 \\
      &= \epsilon
    \end{align*}
    and $\lim_{n\to\infty} A_n(1) = \sum_{k=1}^\infty c_n = \overline{A}$.
  \end{enumerate}
\end{document}
