\documentclass[oneside]{article}

\usepackage{amsfonts}
\usepackage{amsmath}
\usepackage{amssymb}
\usepackage{amsthm}
\usepackage{enumitem}
\usepackage{mathtools}

\allowdisplaybreaks

% Notation shortcuts
\newcommand\abs[1]{\left|#1\right|}
\newcommand\defeq{\overset{\mathrm{def}}{=}}
\newcommand*\Laplace{\mathop{}\!\mathbin\bigtriangleup}

\DeclarePairedDelimiter\ceil{\lceil}{\rceil}
\DeclarePairedDelimiter\floor{\lfloor}{\rfloor}

\DeclareMathOperator{\sgn}{sgn}
\DeclareMathOperator{\atantwo}{atan2}

\newcommand\bbC{\mathbb{C}}
\newcommand\bbR{\mathbb{R}}
\newcommand\bbZ{\mathbb{Z}}

\newtheorem*{lem}{Lemma}
\newtheorem*{cor}{Corollary}

\begin{document}
  \begin{lem}
    Let $f$ be $2\pi$-periodic and differentiable. Then\[
      \int_{-\pi}^\pi f(x)e^{-inx}dx
      = \frac{1}{in}\int_{-\pi}^\pi f'(x)e^{-inx}dx
    \] for all integer $n$.
  \end{lem}

  \begin{proof}
    By parts,
    \[
      \int_{-\pi}^\pi f(x)e^{-inx}dx
      = -\frac{1}{in}\left[f(x)e^{-inx}\right]_{-\pi}^\pi
      + \frac{1}{in}\int_{-\pi}^\pi f'(x)e^{-inx}dx \text{.}
    \] We observe that \[
      \left[f(x)e^{-inx}\right]_{-\pi}^\pi = 0
    \] since $f(x)e^{-inx}$ is $2\pi$-periodic.
  \end{proof}

  Applying the lemma $k$ times, we get \[
    \int_{-\pi}^\pi f(x)e^{-inx}dx
      = \frac{1}{(in)^k}\int_{-\pi}^\pi f^{(k)}(x)e^{-inx}dx \text{.}
  \] Since $f \in C^k$, $f^{(k)}$ is continuous and hence bounded on
  $[-\pi, \pi]$. Let $0 \leq B$ be such that \[
    \abs{f^{(k)}} \leq B\quad\text{for }x \in [-\pi, \pi] \text{.}
  \] Note that $B$ is independent of $n$.

  Then \begin{align*}
    \abs{\int_{-\pi}^\pi f(x)e^{-inx}dx}
      &= \abs{n}^{-k}\abs{\int_{-\pi}^\pi f^{(k)}(x)e^{-inx}dx} \\
      &\leq \abs{n}^{-k}\int_{-\pi}^\pi \abs{f^{(k)}(x)e^{-inx}}dx \\
      &\leq \abs{n}^{-k}\int_{-\pi}^\pi \abs{f^{(k)}(x)}dx \\
      &\leq \abs{n}^{-k}\int_{-\pi}^\pi B\,dx \\
      &= 2\pi \abs{n}^{-k} B \\
  \end{align*}

  Hence, \[
    \abs{\hat f(n)} = \frac{1}{2\pi}\abs{\int_{-\pi}^\pi f(x)e^{-inx}dx}
    \leq \abs{n}^{-k}B
  \] and $\hat f(n) = O(1/\abs{n}^k)$.\qed
\end{document}
