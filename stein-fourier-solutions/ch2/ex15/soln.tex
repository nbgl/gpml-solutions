\documentclass[oneside]{article}

\usepackage{amsfonts}
\usepackage{amsmath}
\usepackage{amssymb}
\usepackage{amsthm}
\usepackage{enumitem}
\usepackage{mathtools}

\allowdisplaybreaks

% Notation shortcuts
\newcommand\abs[1]{\left|#1\right|}
\newcommand\defeq{\overset{\mathrm{def}}{=}}
\newcommand*\Laplace{\mathop{}\!\mathbin\bigtriangleup}

\DeclarePairedDelimiter\ceil{\lceil}{\rceil}
\DeclarePairedDelimiter\floor{\lfloor}{\rfloor}

\DeclareMathOperator{\sgn}{sgn}
\DeclareMathOperator{\atantwo}{atan2}

\newcommand\bbC{\mathbb{C}}
\newcommand\bbR{\mathbb{R}}
\newcommand\bbZ{\mathbb{Z}}

\newtheorem*{lem}{Lemma}
\newtheorem*{cor}{Corollary}

\begin{document}
  The $n$\textsuperscript{th} Dirichlet $D_n$ is defined as\[
    D_n(x) = \sum_{k=-n}^n e^{ikx} \text{.}
  \] This is a geometric sequence, which we can simplify to \[
    D_n(x) = \frac{\omega^{-n} - \omega^{n+1}}{1-\omega} \text{,}
  \] where we let $\omega = e^{ix}$

  $F_N$ is defined as\[
    NF_N(x) = \sum_{n=0}^{N-1}D_n(x)
  \] for $N = 1, 2, \dots$. Then \begin{align*}
    NF_N(x)
    &= \sum_{n=0}^{N-1}\frac{\omega^{-n} - \omega^{n+1}}{1-\omega} \\
    &= \frac{1}{1-\omega} \sum_{n=0}^{N-1}
    \left(\omega^{-n} - \omega^{n+1}\right) \\
    &= \frac{\omega^0 - \omega^1 + \omega^{-1} - \omega^{2} + \omega^{-2} - \omega^{3} + \dots + \omega^{1-N} - \omega^{N}}{1-\omega} \\
    &= \frac{\omega^0 + \omega^{-1} +  \dots + \omega^{1-N}}{1-\omega} - \frac{\omega^{N} + \omega^{N-1} + \dots + \omega^1}{1-\omega} \\
    &= \frac{1 - \omega^{-N}}{(1-\omega)(1-\omega^{-1})}
    - \frac{\omega^N - 1}{(1-\omega)(1-\omega^{-1})} \\
    &= \frac{2 - \omega^N - \omega^{-N}}{2 - \omega - \omega^{-1}} \\
    &= \frac{1 - \cos Nx}{1 - \cos x} \\
    % &= \frac{\cos(Nx/2-Nx/2) - \cos(Nx/2+Nx/2)}{\cos(x/2-x/2) - \cos(x/2+x/2)} \\
    % &= \frac{2\,\sin(Nx/2)\,\sin(Nx/2)}{2\,\sin(x/2)\,\sin(x/2)} \\
    &= \frac{\sin^2(Nx/2)}{\sin^2(x/2)} \text{,}
  \end{align*} where for the last line we use\begin{align*}
    1 - \cos\alpha
    &= \cos(\alpha/2-\alpha/2) - \cos(\alpha/2+\alpha/2) \\
    &= 2\sin(\alpha/2)\sin(\alpha/2) \\
    &= 2\sin^2(\alpha/2) \text{.}
  \end{align*}

  Then \[
    F_N(x) = \frac1N \frac{\sin^2(Nx/2)}{\sin^2(x/2)} \text{.}
  \]\qed
\end{document}
