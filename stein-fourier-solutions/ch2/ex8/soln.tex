\documentclass[oneside]{article}

\usepackage{amsfonts}
\usepackage{amsmath}
\usepackage{amssymb}
\usepackage{amsthm}
\usepackage{enumitem}
\usepackage{mathtools}

\allowdisplaybreaks

% Notation shortcuts
\newcommand\abs[1]{\left|#1\right|}
\newcommand\defeq{\overset{\mathrm{def}}{=}}
\newcommand*\Laplace{\mathop{}\!\mathbin\bigtriangleup}

\DeclareMathOperator{\sgn}{sgn}
\DeclareMathOperator{\atantwo}{atan2}

\newcommand\bbC{\mathbb{C}}
\newcommand\bbR{\mathbb{R}}
\newcommand\bbZ{\mathbb{Z}}

\newtheorem*{lem}{Lemma}

\begin{document}
  The Fourier coefficients of $f$ are\[
    \hat f(n) = \frac{1}{2\pi}\int_{-\pi}^\pi f(x) e^{-inx} dx\text{.}
  \] Observing that $f$ is odd, this becomes\[
    \hat f(n) = \frac{1}{2\pi i}\int_{-\pi}^\pi f(x) \sin nx\,dx
    = \frac{1}{\pi i}\int_0^\pi f(x) \sin nx\,dx \text{.}
  \] For $n=0$, we get $\hat f(n) = 0$. Computing the integral for $n \neq 0$,
  \begin{align*}
    \hat f(n)
    &= \frac{1}{\pi i}\int_0^\pi\left(\frac\pi2-\frac x2\right)\sin nx\,dx \\
    &= \frac{1}{\pi i}\frac{\pi n + \sin \pi n}{2n^2} \\
    &= \frac{1}{2ni}\text{.} &\text{($\sin\pi n = 0$)}
  \end{align*} Thus\[
    f(x) \sim \frac{1}{2i}\sum_{n\neq0} \frac{e^{inx}}{n}
    = \frac{1}{2i} \sum_{n=1}^\infty \frac1n\left(e^{inx} - e^{-inx}\right)
    \text{.}
  \]

  We now show that this series converges for all $x$. For $x=0$, the Fourier
  series converges to $0 = f(0)$ because $\hat f(n)$ is odd. For $x\neq 0$, we
  argue by Dirichlet's test. Let $a_n = 1/n$. We can see that $a_n$ converges to
  $0$ monotonically as $n\to\infty$. Let $b_n(x) = e^{inx} - e^{-inx}$ and
  $B_N(x) = \sum_{n=1}^N b_n(x)$. $B_N(x)$ is the $N$\textsuperscript{th}
  Dirichlet kernel and\[
    B_N(x) = D_N(x) = \frac{\sin((N+\frac12)x)}{\sin(x/2)}\text{.}
  \] by section 1.1, example 4. The denominator $\sin(x/2)$ is constant and the
  numerator is bounded by $\abs{\sin((N+1/2)x)} \leq 1$, so $B_N(x)$ is bounded
  for a given $x$. Hence, by Dirichlet's test,\[
    \sum_{n=1}^\infty a_nb_n
    = \frac{1}{2i} \sum_{n=1}^\infty \frac1n\left(e^{inx} - e^{-inx}\right)
    = \frac{1}{2i}\sum_{n\neq0} \frac{e^{inx}}{n}
  \] converges.

\end{document}
