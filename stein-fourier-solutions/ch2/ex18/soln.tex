\documentclass[paperoneside]{article}

\usepackage{amsfonts}
\usepackage{amsmath}
\usepackage{amssymb}
\usepackage{amsthm}
\usepackage{enumitem}
\usepackage{mathtools}

\allowdisplaybreaks

% Notation shortcuts
\newcommand\abs[1]{{\left|#1\right|}}
\newcommand\defeq{\overset{\mathrm{def}}{=}}
\newcommand*\Laplace{\mathop{}\!\mathbin\bigtriangleup}

\DeclarePairedDelimiter\ceil{\lceil}{\rceil}
\DeclarePairedDelimiter\floor{\lfloor}{\rfloor}

\DeclareMathOperator{\sgn}{sgn}
\DeclareMathOperator{\atantwo}{atan2}

\newcommand\bbC{\mathbb{C}}
\newcommand\bbR{\mathbb{R}}
\newcommand\bbZ{\mathbb{Z}}

\newtheorem*{lem}{Lemma}
\newtheorem*{cor}{Corollary}

\begin{document}
  The Poisson kernel can be expressed as the Fourier series\[
    P_r(\theta) = \sum_{-\infty}^{n=\infty} r^\abs{n} e^{in\theta} \text{.}
  \] We define \[
    u(r, \theta) = \frac{\partial P_r(\theta)}{\partial \theta} \text{.}
  \]

  We can rearrange\[
    P_r(\theta)
    = 1 + \sum_{n=1}^\infty r^n e^{in\theta} + \sum_{n=1}^\infty r^n e^{-in\theta}
    = 1 + \frac{r e^{i\theta}}{1-re^{i\theta}} + \frac{r e^{-i\theta}}{1-re^{-i\theta}} \text{.}
  \]

  Differentiating,\[
    u(r, \theta)
    = \frac{\partial P_r(\theta)}{\partial\theta}
    = \frac{ire^{i\theta}}{\left(1-re^{i\theta}\right)^2}
    - \frac{ir e^{-i\theta}}{\left(1-re^{-i\theta}\right)^2} \text{.}
  \]

  \begin{enumerate}[label=(\roman*)]
    \item We want to show that $\Delta u = 0$ in the unit disc.

    We know that \[
      \Delta u
      = \frac1r\frac\partial{\partial r}r\frac{\partial u}{\partial r}
      + \frac1{r^2}\frac{\partial^2 u}{\partial \theta^2} \text{.}
    \]

    We first find $\partial u/\partial r$. Differentiating w.r.t. $r$,\begin{align*}
      \frac{\partial u}{\partial r}
      &= ie^{i\theta}\frac{(1-re^{i\theta})^2+2re^{i\theta}(1-re^{i\theta})}{(1-re^{i\theta})^4} \\
      &\qquad- ie^{-i\theta}\frac{(1-re^{-i\theta})^2+2re^{-i\theta}(1-re^{-i\theta})}{(1-re^{-i\theta})^4} \\
      &= ie^{i\theta}\frac{1+re^{i\theta}}{(1-re^{i\theta})^3} 
      - ie^{-i\theta}\frac{1+re^{-i\theta}}{(1-re^{-i\theta})^3} \text{.}
    \end{align*} Differentiating again to get $\partial^2 u/\partial r^2$,\begin{align*}
      \frac{\partial^2 u}{\partial r^2}
      &= ie^{i\theta}\frac{(1-re^{i\theta})^3e^{i\theta} + 3e^{i\theta}(1-re^{i\theta})^2(1+re^{i\theta})}{(1-re^{i\theta})^6} \\
      &\qquad- ie^{-i\theta}\frac{(1-re^{-i\theta})^3e^{-i\theta} + 3e^{-i\theta}(1-re^{-i\theta})^2(1+re^{-i\theta})}{(1-re^{-i\theta})^6} \\
      &= 2ie^{2i\theta}\frac{2+re^{i\theta}}{(1-re^{i\theta})^4}
      - 2ie^{-2i\theta}\frac{2+re^{-i\theta}}{(1-re^{-i\theta})^4} \text{.}
    \end{align*}

    We now find $\partial^2 u/\partial \theta^2$. Differentiating twice,
    \begin{align*}
      \frac{\partial u}{\partial\theta}
      &= ir\frac{ie^{i\theta}(1-re^{i\theta})^2 + 2ire^{2i\theta}(1-re^{i\theta})}{(1-re^{i\theta})^4} \\
      &\qquad- ir\frac{-ie^{-i\theta}(1-re^{-i\theta})^2 - 2e^{-i\theta}ire^{-i\theta}(1-re^{-i\theta})}{(1-re^{-i\theta})^4} \\
      &= -r\frac{e^{i\theta} + re^{2i\theta}}{(1-re^{i\theta})^3}
      - r\frac{e^{-i\theta} + re^{-2i\theta}}{(1-re^{-i\theta})^3}
    \end{align*}
    and
    \begin{align*}
      &\frac{\partial^2 u}{\partial\theta^2} \\
      &= -ir\frac{(e^{i\theta} + 2re^{2i\theta})(1-re^{i\theta}) + 3re^{i\theta}(e^{i\theta} + re^{2i\theta})}{(1-re^{i\theta})^4} \\
      &\qquad + ir\frac{(e^{-i\theta} + 2re^{-2i\theta})(1-re^{-i\theta}) + 3re^{-i\theta}(e^{-i\theta} + re^{-2i\theta})}{(1-re^{-i\theta})^4} \\
      &= -ir\frac{e^{i\theta} + 4re^{2i\theta} + r^2e^{3i\theta}}{(1-re^{i\theta})^4} + ir\frac{e^{-i\theta} + 4re^{-2i\theta} + r^2e^{-3i\theta}}{(1-re^{-i\theta})^4} \text{.}
    \end{align*}

    Finally\begin{align*}
      \Delta u
      &= \frac1r \frac\partial{\partial r} r \frac{\partial u}{\partial r} + \frac1{r^2} \frac{\partial^2 u}{\partial \theta^2} \\
      &= \frac1r \frac{\partial u}{\partial r}
      + \frac{\partial^2 u}{\partial r^2}
      + \frac1{r^2} \frac{\partial^2 u}{\partial \theta^2} \\
      &= ie^{i\theta}\frac{1+re^{i\theta}}{r(1-re^{i\theta})^3} 
      - ie^{-i\theta}\frac{1+re^{-i\theta}}{r(1-re^{-i\theta})^3} \\
      &\qquad + 2ie^{2i\theta}\frac{2+re^{i\theta}}{(1-re^{i\theta})^4}
      - 2ie^{-2i\theta}\frac{2+re^{-i\theta}}{(1-re^{-i\theta})^4} \\
      &\qquad -i\frac{e^{i\theta} + 4re^{2i\theta} + r^2e^{3i\theta}}{r(1-re^{i\theta})^4} + i\frac{e^{-i\theta} + 4re^{-2i\theta} + r^2e^{-3i\theta}}{r(1-re^{-i\theta})^4} \\
      &= \frac{ie^{i\theta}}{r(1-re^{i\theta})^4}\Big[
      (1+re^{i\theta})(1 - re^{i\theta}) \\
      &\qquad\qquad+ 2re^{i\theta}(2+re^{i\theta})
      - (1 + 4re^{i\theta} + r^2e^{2i\theta}) \Big] \\
      &\qquad- \frac{ie^{-i\theta}}{r(1-re^{-i\theta})^4}\Big[
      (1+re^{-i\theta})(1 - re^{-i\theta}) \\
      &\qquad\qquad+ 2re^{-i\theta}(2+re^{-i\theta})
      - (1 + 4re^{-i\theta} + r^2e^{-2i\theta}) \Big] \\
      &= \frac{ie^{i\theta}}{r(1-re^{i\theta})^4}\Big[
      1-r^2e^{2i\theta} + 4re^{i\theta}+2r^2e^{2i\theta}
      - 1 - 4re^{i\theta} - r^2e^{2i\theta}) \Big] \\
      &\qquad- \frac{ie^{-i\theta}}{r(1-re^{-i\theta})^4}\Big[
      1-r^2e^{-2i\theta} + 4re^{-i\theta}+2r^2e^{-2i\theta} \\
      &\qquad\qquad- 1 - 4re^{-i\theta} - r^2e^{-2i\theta} \Big] \\
      &= 0 \text{.}
    \end{align*}\qed

    \item Recall that \[
      u(r, \theta)
      = \frac{ire^{i\theta}}{(1 - re^{i\theta})^2}
      - \frac{ire^{-i\theta}}{(1 - re^{-i\theta})^2} \text{.}
    \] 

    We argue by cases on $\theta$. \begin{itemize}[leftmargin=43pt]
      \item[($e^{i\theta} \neq 1$)] Taking the limit,\[
          \lim_{r\to1} u(r, \theta)
          = \frac{ie^{i\theta}}{(1-e^{i\theta})^2}
          - \frac{ie^{-i\theta}}{(1-e^{-i\theta})^2} \text{.}
        \] Since $e^{i\theta} \neq 1$ by assumption, the denominators are non-zero and this limit is well-defined. Simplifying,\[
          \lim_{r\to1} u(r, \theta)
          = \frac{0}{(1-e^{i\theta})^2(1-e^{-i\theta})^2} = 0
           \text{.}
        \]

      \item[($e^{i\theta} = 1$)] Since $e^{i\theta} = 1$ by assumption, $u$ simplifies to \[
        u(r, \theta)
        = \frac{ir}{(1 - r)^2}
        - \frac{ir}{(1 - r)^2} = 0 \text{.}
      \] for all $0 \leq r < 1$. Hence $\lim_{r\to1} u(r, \theta) = 0$.\qed
      \end{itemize}
  \end{enumerate}
\end{document}
