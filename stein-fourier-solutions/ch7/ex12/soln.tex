\documentclass[oneside]{article}

\usepackage{amsfonts}
\usepackage{amsmath}
\usepackage{amssymb}
\usepackage{amsthm}
\usepackage{enumitem}
\usepackage{mathtools}
\usepackage{cleveref}

\allowdisplaybreaks

% Notation shortcuts
\newcommand\abs[1]{\left|#1\right|}
\newcommand\norm[1]{\left\|#1\right\|}
\newcommand\defeq{\overset{\mathrm{def}}{=}}
\newcommand*\Laplace{\mathop{}\!\mathbin\bigtriangleup}

\DeclarePairedDelimiter\ceil{\lceil}{\rceil}
\DeclarePairedDelimiter\floor{\lfloor}{\rfloor}

\DeclareMathOperator{\sgn}{sgn}
\DeclareMathOperator{\atantwo}{atan2}

\newcommand\bbC{\mathbb{C}}
\newcommand\bbR{\mathbb{R}}
\newcommand\bbQ{\mathbb{Q}}
\newcommand\bbZ{\mathbb{Z}}

\newtheorem*{lem}{Lemma}
\newtheorem*{cor}{Corollary}

\renewcommand{\thefootnote}{[\arabic{footnote}]}

\begin{document}
  \begin{proof}
    Observe that $e(1) = e(1^2) = (e(1))^2$, so $e(1)$ is either $0$ or $1$. In the former case, for all $x \in G$, $e(x) = e(1 \cdot x) = e(1)e(x) = 0$, so $e$ is identically $0$.

    Otherwise, pick any $y \in G$ and consider the sequence $1, y, y^2, y^3, \dots$. Since $G$ is finite, the sequence has a cycle and we can find $0 \leq i < N$ and the smallest $0 < n$ such that $y^i = y^{i+n}$, implying that $y^n = 1$ and the cycle has length $n$. Therefore, $Y$ generates the set $y = \{1, y, y^2, \dots, y^{n-1}\}$, which is a subgroup of $G$ of order $n$.

    Define an equivalence relation for all $x, x' \in G$ by $x \sim x'$ if there exists $y \in Y$ such that $x = yx'$. Reflexivity and symmetry are easy to see. To show transitivity, pick $x, x', x'' \in G$ such that $x \sim x'$ and $x' \sim x''$. Then there exist $y, y' \in Y$ such that $x = yx'$ and $x' = y'x''$, implying that $x=(yy')x''$.

    We now show that each equivalence class has exactly $n$ members. For all $x \in X$ and $y, y' \in Y$ we have that $yx = y'x$ implies $y = y'$. Since $1, y, \dots, y^{n-1}$ are unique, the equivalence class of $x$ is $\{x, yx, y^2x, \dots, y^{n-1}x\}$, which has cardinality $n$.

    We have thus partitioned $G$ into equivalence classes of cardinality $n$, which is only possible if $n|N$. Then $(e(y))^N = e(y^N) = e(y^{N \mod n}) = e(1) = 1$. Thus, $e(y)$ is an $N$\textsuperscript{th} root of unity and has the form \[
      e(y) = \exp\left(2\pi i \frac{r}{N}\right)
    \] for some $r \in \{0, 1, \dots, N-1\}$.\end{proof}
\end{document}
