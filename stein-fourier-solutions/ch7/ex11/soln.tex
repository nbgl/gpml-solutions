\documentclass[oneside]{article}

\usepackage{amsfonts}
\usepackage{amsmath}
\usepackage{amssymb}
\usepackage{amsthm}
\usepackage{enumitem}
\usepackage{mathtools}
\usepackage{cleveref}

\allowdisplaybreaks

% Notation shortcuts
\newcommand\abs[1]{\left|#1\right|}
\newcommand\norm[1]{\left\|#1\right\|}
\newcommand\defeq{\overset{\mathrm{def}}{=}}
\newcommand*\Laplace{\mathop{}\!\mathbin\bigtriangleup}

\DeclarePairedDelimiter\ceil{\lceil}{\rceil}
\DeclarePairedDelimiter\floor{\lfloor}{\rfloor}

\DeclareMathOperator{\sgn}{sgn}
\DeclareMathOperator{\atantwo}{atan2}

\newcommand\bbC{\mathbb{C}}
\newcommand\bbR{\mathbb{R}}
\newcommand\bbQ{\mathbb{Q}}
\newcommand\bbZ{\mathbb{Z}}

\newtheorem*{lem}{Lemma}
\newtheorem*{cor}{Corollary}

\renewcommand{\thefootnote}{[\arabic{footnote}]}

\begin{document}
  \begin{center}
    \begin{tabular} {r | c c}
      $\bbZ^*(3)$ & $1$ & $2$ \\
      \hline
      $1$ & $1$ & $2$ \\
      $2$ & $2$ & $1$
    \end{tabular}\\[20pt]

    \begin{tabular} {r | c c}
      $\bbZ^*(4)$ & $1$ & $3$ \\
      \hline
      $1$ & $1$ & $3$ \\
      $3$ & $3$ & $1$
    \end{tabular}\\[20pt]

    \begin{tabular} {r | c c c c}
      $\bbZ^*(5)$ & $1$ & $2$ & $3$ & $4$ \\
      \hline
      $1$ & $1$ & $2$ & $3$ & $4$ \\
      $2$ & $2$ & $4$ & $1$ & $3$ \\
      $3$ & $3$ & $1$ & $4$ & $2$ \\
      $4$ & $4$ & $3$ & $2$ & $1$ \\
    \end{tabular}\\[20pt]

    \begin{tabular} {r | c c}
      $\bbZ^*(6)$ & $1$ & $5$ \\
      \hline
      $1$ & $1$ & $5$ \\
      $5$ & $5$ & $1$
    \end{tabular}\\[20pt]

    \begin{tabular} {r | c c c c}
      $\bbZ^*(8)$ & $1$ & $3$ & $5$ & $7$ \\
      \hline
      $1$ & $1$ & $3$ & $5$ & $7$ \\
      $3$ & $3$ & $1$ & $7$ & $5$ \\
      $5$ & $5$ & $7$ & $1$ & $3$ \\
      $7$ & $7$ & $5$ & $3$ & $1$ \\
    \end{tabular}\\[20pt]

    \begin{tabular} {r | c c c c c c}
      $\bbZ^*(9)$ & $1$ & $2$ & $4$ & $5$ & $7$ & $8$ \\
      \hline
      $1$ & $1$ & $2$ & $4$ & $5$ & $7$ & $8$ \\
      $2$ & $2$ & $4$ & $8$ & $1$ & $5$ & $7$ \\
      $4$ & $4$ & $8$ & $7$ & $2$ & $1$ & $5$ \\
      $5$ & $5$ & $1$ & $2$ & $7$ & $8$ & $4$ \\
      $7$ & $7$ & $5$ & $1$ & $8$ & $4$ & $2$ \\
      $8$ & $8$ & $7$ & $5$ & $4$ & $2$ & $1$ \\
    \end{tabular}\\[20pt]
  \end{center}

  $\bbZ^*(3)$, $\bbZ^*(4)$, $\bbZ^*(5)$, $\bbZ^*(6)$, and $\bbZ^*(9)$ are all cyclic, generated by $2$, $3$, $2$, $5$, and $2$, respectively. $\bbZ^*(8)$ is not cyclic: every element is its own inverse so there is no element of order greater than $2$.

\end{document}
