\documentclass[oneside]{article}

\usepackage[pass]{geometry}
\newlength\DX
\DX=3.5in
\paperwidth=\dimexpr\paperwidth-\DX\relax
\hoffset=\dimexpr\hoffset-.5\DX\relax
\newlength\DY
\DY=2.8in
\paperheight=\dimexpr\paperheight-\DY\relax
\voffset=\dimexpr\voffset-.5\DY-.5\footskip\relax

\usepackage{amsfonts}
\usepackage{amsmath}
\usepackage{amssymb}
\usepackage{amsthm}
\usepackage{enumitem}
\usepackage{mathtools}
\usepackage{cleveref}

\allowdisplaybreaks

% Notation shortcuts
\newcommand\abs[1]{\left|#1\right|}
\newcommand\norm[1]{\left\|#1\right\|}
\newcommand\defeq{\overset{\mathrm{def}}{=}}
\newcommand*\Laplace{\mathop{}\!\mathbin\bigtriangleup}

\DeclarePairedDelimiter\ceil{\lceil}{\rceil}
\DeclarePairedDelimiter\floor{\lfloor}{\rfloor}

\DeclareMathOperator{\sgn}{sgn}
\DeclareMathOperator{\atantwo}{atan2}

\newcommand\bbC{\mathbb{C}}
\newcommand\bbR{\mathbb{R}}
\newcommand\bbQ{\mathbb{Q}}
\newcommand\bbZ{\mathbb{Z}}

\newtheorem*{lem}{Lemma}
\newtheorem*{cor}{Corollary}

\renewcommand{\thefootnote}{[\arabic{footnote}]}

\begin{document}
  \begin{proof}
    To prove completeness of $\bbC^d$, consider $\bbR^{2d}$ and identify every\[
      (x_1+i y_1, x_2 +iy_2, \dots, x_d + iy_d) \in \bbC^d
    \] with \[
      (x_1, y_1, x_2, y_2, \dots, x_d, y_d) \in \bbR^{2d} \text{.}
    \] This transformation is isomorphic under addition and preserves the norm, so it is an isometry.

    Hence, it is sufficient to show the completeness of $\bbR^d$ for all $d$. Let $\{v_n\}_{n=1}^\infty \subset \bbR^d$ be a Cauchy sequence. For all $n$, denote\[
      v_n = (v_{n,1}, \dots, v_{n,d}) \text{.}
    \]

    For every dimension $k=1, \dots, d$ consider the sequence $\{v_{n,k}\}_{n=1}^\infty$. It is Cauchy, since for all $n, m$, $\abs{v_{n,k} - v_{m,k}}$ is bounded by $\norm{v_n - v_m}$, so it converges to some $\ell_k \in \bbR$ by completeness of $\bbR$.

    Let $\ell \in \bbR^d = (\ell_1, \dots, \ell_d)$. We decompose the distance as\[
      \norm{v_n - \ell}^2 = \sum_{k=1}^d (v_{n,k} - \ell_k)^2
    \] and observe that the RHS goes to 0 as $n \to \infty$ by convergence of $\{v_{n,k}\}_{n=1}^\infty$. Hence, $v_n$ converges to $\ell$ as $n \to \infty$.
  \end{proof}
\end{document}
