\documentclass[oneside]{article}

\usepackage{amsfonts}
\usepackage{amsmath}
\usepackage{amssymb}
\usepackage{amsthm}
\usepackage{enumitem}
\usepackage{mathtools}
\usepackage{cleveref}

\allowdisplaybreaks

% Notation shortcuts
\newcommand\abs[1]{\left|#1\right|}
\newcommand\norm[1]{\left\|#1\right\|}
\newcommand\defeq{\overset{\mathrm{def}}{=}}
\newcommand*\Laplace{\mathop{}\!\mathbin\bigtriangleup}

\DeclarePairedDelimiter\ceil{\lceil}{\rceil}
\DeclarePairedDelimiter\floor{\lfloor}{\rfloor}

\DeclareMathOperator{\sgn}{sgn}
\DeclareMathOperator{\atantwo}{atan2}

\newcommand\bbC{\mathbb{C}}
\newcommand\bbR{\mathbb{R}}
\newcommand\bbQ{\mathbb{Q}}
\newcommand\bbZ{\mathbb{Z}}

\newtheorem*{lem}{Lemma}
\newtheorem*{cor}{Corollary}

\renewcommand{\thefootnote}{[\arabic{footnote}]}

\begin{document}
  \begin{proof}
    For all $n, m$ with $n < m$, observe that \[
      f_m(\theta) - f_n(\theta)
      = \begin{cases}
        0 &\text{for }0 \leq \theta \leq 1/m \\
        \log(1/\theta) &\text{for }1/m < \theta \leq 1/n \\
        0 &\text{for }1/n < \theta \leq 2\pi \text{.}
      \end{cases}
    \]

    Then \[
      \norm{f_m - f_n}^2
      = \frac1{2\pi}\int_{1/m}^{1/n} (\log(1/\theta))^2\;d\theta \text{.}
    \]

    We solve the integral by observing that \[
      \frac{d}{d\theta}\Big( \theta(\log(1/\theta))^2 - 2 \theta \log (1/\theta) + 2 \theta \Big) = (\log (1/\theta))^2 \text{,}
    \] so \begin{align*}
      \norm{f_m - f_n}^2
      &= \frac{(\log n)^2 + 2 \log n + 2}{n} - \frac{(\log m)^2 + 2 \log m + 2}{m} \\
      &< \frac{(\log n)^2 + 2 \log n + 2}{n} \text{.}
    \end{align*} The denominator grows faster than the numerator, so \[
      \lim_{n\to \infty} \frac{(\log n)^2 + 2 \log n + 2}{n} = 0 \text{.}
    \] Recalling that $m > n$, this implies that $\{f_n\}_{n=1}^\infty$ is Cauchy.
  \end{proof}

\end{document}
