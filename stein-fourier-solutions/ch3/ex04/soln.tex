\documentclass[oneside]{article}

\usepackage{amsfonts}
\usepackage{amsmath}
\usepackage{amssymb}
\usepackage{amsthm}
\usepackage{enumitem}
\usepackage{mathtools}
\usepackage{cleveref}

\allowdisplaybreaks

% Notation shortcuts
\newcommand\abs[1]{\left|#1\right|}
\newcommand\norm[1]{\left\|#1\right\|}
\newcommand\defeq{\overset{\mathrm{def}}{=}}
\newcommand*\Laplace{\mathop{}\!\mathbin\bigtriangleup}

\DeclarePairedDelimiter\ceil{\lceil}{\rceil}
\DeclarePairedDelimiter\floor{\lfloor}{\rfloor}

\DeclareMathOperator{\sgn}{sgn}
\DeclareMathOperator{\atantwo}{atan2}

\newcommand\bbC{\mathbb{C}}
\newcommand\bbR{\mathbb{R}}
\newcommand\bbQ{\mathbb{Q}}
\newcommand\bbZ{\mathbb{Z}}
\newcommand\calR{\mathcal{R}}

\newtheorem*{lem}{Lemma}
\newtheorem*{cor}{Corollary}

\renewcommand{\thefootnote}{[\arabic{footnote}]}

\begin{document}
  \begin{enumerate}[label=(\alph*)]
    \item
      \begin{proof}
        Consider the function $f : [0, 2\pi] \to \bbR$ defined by \[
            f(x) = \begin{cases}
              0 &\text{if }x \neq \pi \\
              1 &\text{if }x = \pi \text{.}
            \end{cases}
        \] Then $\abs{f(x)}^2 = f(x)$. We integrate it by considering the sequence of partitions $\{P_n\}_{n=1}^\infty$ with $P_n = \{0, \pi - 1/n, \pi + 1/n, 2\pi\}$. The lower sum is $L_n = 0$ and the upper sum $U_n = 2/n$ converges to $0$ as $n\to\infty$. Hence, $f \in \calR$ and $\norm{f} = 0$.
      \end{proof}

    \item
      \begin{proof}
        We argue by contrapositive. Consider an integrable function $f : [0, 2\pi] \to \bbR$ that is continuous at some $x_0 \in [0, 2\pi]$ with $f(x_0) \neq 0$.

        Since composition with a continuous functions preserves continuity, $g(x) \coloneqq \abs{f(x)}^2 \geq 0$ is also continuous at $x_0$ and $g(x_0) > 0$. Let $\epsilon = g(x_0) / 2 > 0$. There exists some $\delta > 0$ such that $g(x_0) - \epsilon < g(x)$ for all $x_0 - \delta < x < x_0 + \delta$. Hence, we've established a lower bound of $g(x_0) - \epsilon = g(x_0) / 2 > 0$ for $g$ in the interval $(x_0 - \delta, x_0 + \delta)$. Then \[
          \norm{f}
          = \int_0^{2\pi} g(x)\; dx
          \geq \int_{x_0 - \delta}^{x_0 + \delta} g(x)\; dx
          \geq \delta g(x_0)
          > 0 \text{.}
        \]
      \end{proof}

    \item
      \begin{proof}
        We argue by contrapositive. Consider an integrable function $f : [0, 2\pi] \to \bbR$ such that $\norm{f} > 0$.

        Let $g(x) = \abs{f(x)}^2$ and observe that $2\pi\norm{f}^2 = \int_0^{2\pi}g(x)\;dx > 0$. There exists a partition $P = \{0 = x_0 < \dots < x_N = 2\pi\}$ such that the lower sum \[
          L = \sum_{n=1}^N \big[ \inf_{x \in [x_{n-1}, x_n]} g(x) \big] (x_n - x_{n-1})
        \] is positive since the lower sum converges to $\norm{f} > 0$ with sufficiently fine partitions. Every element of the sum is non-negative, so there exists $n$ such that $\inf_{x \in [x_{n-1}, x_n]} g(x) > 0$.

        Since $f$ is integrable, the set of its discontinuities has measure $0$. The interval $[x_{n-1}, x_n]$ has measure $x_n - x_{n-1} > 0$, so $f$ cannot be discontinuous in that entire interval. There exists $x_{n-1} \leq y \leq x_n$ such that $f$ is continuous at $y$.

        By our bound on $g$, \[
          g(y) \geq \inf_{x \in [x_{n-1}, x_n]} g(x) > 0 \text{.}
        \] Then $f(y) \neq 0$, so $f$ does not vanish at all points of continuity.
      \end{proof}

  \end{enumerate}
\end{document}
