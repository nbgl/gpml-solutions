\documentclass[oneside]{article}

\usepackage{amsfonts}
\usepackage{amsmath}
\usepackage{amssymb}
\usepackage{amsthm}
\usepackage{enumitem}
\usepackage{mathtools}

\allowdisplaybreaks

% Notation shortcuts
\newcommand\abs[1]{\left|#1\right|}
\newcommand\defeq{\overset{\mathrm{def}}{=}}
\newcommand*\Laplace{\mathop{}\!\mathbin\bigtriangleup}

\DeclareMathOperator{\sgn}{sgn}
\DeclareMathOperator{\atantwo}{atan2}

\newcommand\bbC{\mathbb{C}}
\newcommand\bbR{\mathbb{R}}
\newcommand\bbZ{\mathbb{Z}}

\begin{document}
Let $F : (0, \infty) \to \bbR$ twice differentiable such that \[
  r^2F''(x) + rF'(r) - n^2F(r) = 0
\] for some $n \in \bbZ$.

Let $g(r) = F(r)/r^n$. Observe that the denominator is never zero on the domain
of $F$. Then \begin{align*}
  F(r) &= r^ng(r)\text{,} \\
  F'(r) &= nr^{n-1}g(r) + r^ng'(r)\text{,} \\
  F''(r) &= n(n-1)r^{n-2}g(r) + nr^{n-1}g'(r) + nr^{n-1}g'(r) + r^ng''(r)\text{,} \\
  &= n(n-1)r^{n-2}g(r) + 2nr^{n-1}g'(r) + r^ng''(r)\text{.} \\
\end{align*}

Substituting back\begin{align*}
&r^2F''(x) + rF'(r) - n^2F(r) \\
&= r^2\left(n(n-1)r^{n-2}g(r) + 2nr^{n-1}g'(r) + r^ng''(r)\right) \\
&\qquad+ r\left(nr^{n-1}g(r) + r^ng'(r)\right) \\
&\qquad- n^2r^ng(r) \\
&= n(n-1)r^ng(r) + 2nr^{n+1}g'(r) + r^{n+2}g''(r) \\
&\qquad+ nr^ng(r) + r^{n+1}g'(r) \\
&\qquad- n^2r^ng(r) \\
&= (2n + 1)r^{n+1}g'(r) + r^{n+2}g''(r) = 0\text{,}
\end{align*} so\[
  (2n + 1)g'(r) + rg''(r) = 0\text{.}
\]

Integrating by parts,\begin{align*}
  \int g'(r)dr &= g(r) + \mathrm{const}\text{,} \\
  \int rg''(r)dr &= rg'(r) - \int g'(r)dr + \mathrm{const} \\
  &= rg'(r) - g(r) + \mathrm{const} \text{.}
\end{align*}

Hence\[
  (2n + 1)g(r) + rg'(r) - g(r) = rg'(r) + 2ng(r) = c
\] for some constant $c$.

For notational convenience, let $y = g(r)$. Then $g'(r) = dy/dr$ and \[
  r\frac{dy}{dr} + 2ny = c \text{.}
\]

This is separable as \[
  \frac{dr}{r} = \frac{dy}{c-2ny} \text{.}
\]

We now argue by cases: either $n = 0$ or $n \neq 0$.
\begin{itemize}[leftmargin=52pt]
\item[($n = 0$)] We have\[
  \frac{dr}{r} = \frac{dy}{c}
\] Integrating,\[
  \log r = \frac{1}{c}y + \mathrm{const}\text{,}
\] so \[
  g(r) = y = c \log r + d
\] for some constant $d$. Then \[
  F(r) = r^0g(r) = c \log r + d \text{,}
\]so $F$ is a linear combination of $\log r$ and $1$.

\item[($n \neq 0$)]
Integrating,\[
  \log r = -\frac{1}{2n}\log\abs{c-2ny} + \mathrm{const}\text{,}
\] so \[
   \log\abs{c-2ny} = -2n\log r + \mathrm{const}
\] and \[
  2ny - c = dr^{-2n}
\] for some $d$. Hence,
\[
  g(r) = y = \frac{d}{2n}r^{-2n} + \frac{c}{2n} \text{.}
\] Finally,\[
  F(r) = r^ng(r) = \frac{d}{2n}r^{-n} + \frac{c}{2n}r^n\text{,}
\] so $F$ is a linear combinatio of $r^{-n}$ and $r^n$ as desired.\qed
\end{itemize}

\end{document}
