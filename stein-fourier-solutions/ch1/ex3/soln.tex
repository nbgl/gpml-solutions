\documentclass[oneside]{article}

\usepackage{amsfonts}
\usepackage{amsmath}
\usepackage{amssymb}
\usepackage{amsthm}
\usepackage{enumitem}
\usepackage{mathtools}

\allowdisplaybreaks

% Notation shortcuts
\newcommand\abs[1]{\left|#1\right|}
\newcommand\defeq{\overset{\mathrm{def}}{=}}

\newcommand\bbC{\mathbb{C}}
\newcommand\bbR{\mathbb{R}}

\begin{document}
  \begin{enumerate}[label=\textbf{(\alph*)}]
    \item Assume $\{w_n\}_{n=1}^\infty$ converges. Let $w$ and $w'$ be its
      limits. We show that they are equal.

      Observe that for all $n$, \begin{align*}
        \abs{(w_n - w) - (w_n - w')}
        &\leq \abs{w_n - w} + \abs{-(w_n - w')} \\
        &= \abs{w_n - w} + \abs{w_n - w'}\text,
      \end{align*} where the first line is by the triangle inequality.

      Since \[
        \lim_{n\to\infty}\abs{w_n - w} = 0
        \text{ and }\lim_{n\to\infty}\abs{w_n - w'} = 0
      \] by assumption,\[
        \lim_{n\to\infty}\abs{(w_n - w) - (w_n - w')} = 0
      \] by the squeeze theorem (here, the fact that the modulus is non-negative
      places a lower bound on the limit).

      Observe that $(w_n - w) - (w_n - w') = w' - w$, so this sequence is
      constant. Since its limit is $0$, we have that $w' - w = 0$. \qed

    \item Both directions use the observation that for all complex numbers
      $z = x + iy$ with $x, y \in \bbR$,
      \begin{align} \label{eq:b-lemma}
        \abs{z} &= \sqrt{x^2 + y^2} \nonumber\\
        &\geq \sqrt{x^2} \\
        &= \abs{x} \text{.} \nonumber
      \end{align} A similar argument shows that $\abs{z} \geq \abs{y}$.

      \begin{itemize}[leftmargin=34pt]
      \item[($\implies$)] Let $\{w_n\}_{n=1}^\infty \subset \bbC$ be a
        convergent. We show that it is Cauchy.

      Decompose $w = t + is$, $w_n = t_n + is_n$ for $t, s, t_n, s_n \in \bbR$.
      Dealing with the real and imaginary parts of $w_n - w$ separately we have
      \[
        \lim_{n\to\infty} \abs{t_n - t} \to 0
        \text{ and } \lim_{n\to\infty} \abs{s_n - s} \to 0
      \] by \eqref{eq:b-lemma} and the squeeze theorem.

      Then $\{t_n\}_{n=1}^\infty$ and $\{s_n\}_{n=1}^\infty$ converge, so they
      are Cauchy.

      Pick an arbitrary $\epsilon > 0$. Find $N$ such that
      $\abs{t_n - t_m} < \epsilon/2$ and $\abs{s_n - s_m} < \epsilon/2$
      whenever $n, m > N$. Then \begin{align*}
        \abs{w_n - w_n} &= \abs{(t_n - t_m) + i(s_n - s_m)} \\
        &\leq \abs{t_n - t_m} + \abs{s_n - s_m} \\
        &< \epsilon
      \end{align*} by the triangle inequality whenever $n, m > N$.

      \item[($\impliedby$)] Let $\{w_n\}_{n=1} \subset \bbC$ be Cauchy. We show
        that it is convergent.

        Pick an arbitrary $\epsilon > 0$. Then there exists a positive integer
        $N$ such that $\abs{w_n - w_m} < \epsilon$ for all $n, m > N$. Decompose
        $w_n = t_n + is_n$ for $t_n, s_n \in \bbR$, and decompose $w_m$
        similarly. Then $\abs{t_n - t_m} \leq \abs{w_n - w_m} < \epsilon$
        $\abs{s_n - s_m} \leq \abs{w_n - w_m} < \epsilon$ by \eqref{eq:b-lemma}.
        Thus $\{t_n\}_{n=0}^\infty$ and $\{s_n\}_{n=0}^\infty$ are Cauchy. It
        follows that they converge.

        Let $t$ and $s$ be the limits of $\{t_n\}_{n=0}^\infty$ and
        $\{s_n\}_{n=0}^\infty$, respectively. Define $w = t + is$.

        We have $\abs{w_n - w} \leq \abs{t_n - t} + \abs{s_n - s}$ by the
        triangle inequality. We also have $\lim_{n\to\infty}\abs{t_n - t} = 0$
        and $\lim_{n\to\infty}\abs{s_n - s} = 0$, which implies
        $\lim_{n\to\infty}(\abs{t_n - t} + \abs{s_n - s}) = 0$. Then by the
        squeeze theorem
        \[
          \lim_{n\to\infty}\abs{w_n - w} = 0
        \] and $\{w_n\}_{n=1} \subset \bbC$ converges.\qed
    \end{itemize}

    \item Let $\{a_n\}_{n=1}^\infty$ be a sequence of non-negative reals such
      that $\sum_{n=1}^\infty a_n$ converges. Let
      $\{z_n\}_{n=1}^\infty \subset \bbC$ be a sequence satisfying
      $\abs{z_n} < a_n$ for all $n$. We show that $\sum_{n=1}^\infty z_n$
      converges.

      Define $S_N = \sum_{n=1}^N z_n$. Our goal is to show that
      $\{S_N\}_{n=1}^\infty$ converges. By (b) it suffices to show that it is
      Cauchy.

      Let $A_N = \sum_{n=1}^N a_n$. By assumption, the sequence formed by these
      partial sums converges, so it is Cauchy.

      Pick an arbitrary $\epsilon > 0$.

      Then there exists a positive integer $M$ such that for all $N, N' > M$,\[
        \abs{A_N - A_{N'}} < \epsilon\text{.}
      \]

      W.l.o.g, assume $N > N'$. Observe that \begin{align*}
        &\abs{A_N - A_{N'}} < \epsilon \\
        &\iff \abs{\sum_{n=1}^N a_n - \sum_{n=1}^{N'} a_n} < \epsilon \\
        &\iff \abs{\sum_{n=N'+1}^N a_n} < \epsilon \\
        &\iff \sum_{n=N'+1}^N a_n < \epsilon
          & \text{($a_n \geq 0 \;\forall n$)}\\
        &\iff \sum_{n=N'+1}^N \abs{z_n} < \epsilon \\
        &\implies \abs{\sum_{n=N'+1}^N z_n} < \epsilon
          & \text{(triangle ineq.)}\\
        &\iff \abs{\sum_{n=1}^N z_n - \sum_{n=1}^{N'} z_n} < \epsilon \\
        &\iff \abs{S_N - S_{N'}} < \epsilon\text{.}
      \end{align*}

      So $\{S_N\}_{n=1}^\infty$ is Cauchy, implying that $\sum_{n=1}^\infty z_n$
      converges.
      \qed

  \end{enumerate}
\end{document}
