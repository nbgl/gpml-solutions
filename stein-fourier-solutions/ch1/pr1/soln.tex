\documentclass[oneside]{article}

\usepackage{amsfonts}
\usepackage{amsmath}
\usepackage{amssymb}
\usepackage{amsthm}
\usepackage{enumitem}
\usepackage{mathtools}

\allowdisplaybreaks

% Notation shortcuts
\newcommand\abs[1]{\left|#1\right|}
\newcommand\defeq{\overset{\mathrm{def}}{=}}
\newcommand*\Laplace{\mathop{}\!\mathbin\bigtriangleup}

\DeclareMathOperator{\sgn}{sgn}
\DeclareMathOperator{\atantwo}{atan2}

\newcommand\bbC{\mathbb{C}}
\newcommand\bbR{\mathbb{R}}
\newcommand\bbZ{\mathbb{Z}}

\newtheorem*{lem}{Lemma}

\begin{document}
\begin{proof}

Let $u_k$ be such that $u_k(x, 0) = A_k\sin kx$, $u_k(x, 1) = B_k\sin kx$,
$u_k(0, y) = 0$, $u_k(1, y) = 0$, and $\Laplace u = 0$.

We want to solve for $u_k$. Using separation of variables, we write
$u_k(x, y) = F(x)G(y)$. The Laplacian becomes \begin{align*}
  \Laplace u_k
  &= \frac{\partial^2F(x)G(y)}{\partial x^2}
     + \frac{\partial^2F(x)G(y)}{\partial y^2} \\
  &= F''(x)G(y) + F(x)G''(y) = 0 \text{.}
\end{align*} Thus, we look for solutions of the form\[
  \frac{F''(x)}{F(x)} = - \frac{G''(y)}{G(y)} \text{.}
\] Since those sides depend on different variables, they must be equal to some
constant, which we will call $\lambda$. Then \[
  F''(x) - \lambda F(x) = 0 \text{ and } G''(y) + \lambda G(y) = 0 \text{.}
\]

By our definition, $u_k(x, 0) = F(x)G(0) = A_k\sin kx$ and
$u(x, 1) = F(x)G(1) = B_k\sin kx$. Then $F(x) = a\sin kx$ for some $a$ and
$\lambda = -k^2$. By the lemma, $G(y) = \alpha\cosh ky - \beta\sinh ky$ for
some $\alpha, \beta \in \bbR$.

When $y = 0$, we have $\alpha F(x) = A_k\sin kx$, so\[
  \alpha F(x)\cosh ky = A_k\sin kx \cosh ky \text{.}
\] Similarly, when $y=1$,
$\alpha F(x) \cosh k - \beta F(x) \sinh k = B_k \sin kx$, implying that\[
  \beta F(x) \sinh ky = \frac{A_k \cosh k - B_k}{\sinh k}\sin kx\sinh ky \text{.}
\] Simplifying,\begin{align*}
  F(x)G(y) &= \left(A_k \cosh ky
                    - \frac{A_k \cosh k - B_k}{\sinh k}\sinh ky\right)\sin kx \\
  &= \left(A_k\frac{\sinh k \cosh ky - \sinh ky\cosh k}{\sinh k} + B_k\frac{\sinh ky}{\sinh k}\right)\sin kx \\
  &= \left(A_k\frac{\sinh k(1-y)}{\sinh k} + B_k\frac{\sinh ky}{\sinh k}\right)
    \sin kx \text{,}
\end{align*} where in the last line we use \begin{align*}
  &4\sinh k \cosh ky - 4\sinh ky \cosh k \\
  &= \left(e^k-e^{-k}\right)\left(e^{ky}+e^{-ky}\right)
    - \left(e^{ky}-e^{-ky}\right)\left(e^k+e^{-k}\right) \\
  &= 2 e^{k - ky} - 2e^{ky - y} = 4 \sinh k(1-y) \text{.}
\end{align*}

Define \[
  u = \sum_{k=1}^\infty u_k
    = \sum_{k=1}^\infty \left(A_k\frac{\sinh k(1-y)}{\sinh k}
           + B_k\frac{\sinh ky}{\sinh k}\right)\sin kx \text{.}
\] Define also\[
  f_0(x) = \sum_{k=1}^\infty A_k\sin kx\text{ and }
  f_1(x) = \sum_{k=1}^\infty B_k\sin kx\text{.}
\] Then $u(x, 0) = \sum_{k=1}^\infty u_k (x, 0) = f_0(x)$, and similarly
$u(x, 1) = f_1(x)$, $u(0, y) = 0$, and $u(1, y) = 0$. Finally, by the linearity
of the Laplacian, $\Laplace u = 0$.\end{proof}

\begin{lem} Let $f$ be a twice continuously differentiable function on $\bbR$
such that $f''(t) - c^2f(t) = 0$. Then all solutions for $f$ have the form\[
  f(t) = a\cosh ct - b\sinh ct \text{.}
\]
\end{lem}

\begin{proof}
Let $g(t) = f(t)\cosh ct - c^{-1} f'(t) \sinh ct$ and
$h(t) = f(t)\sinh ct - c^{-1} f'(t) \cosh ct$. Observe that these once
differentiable. Differentiating,
  \begin{align*}
    g'(t)
    &= f'(t)\cosh ct + cf(t)\sinh ct - c^{-1} f''(t) \sinh ct - f'(t) \cosh ct \\
    &= cf(t)\sinh ct - c^{-1} f''(t) \sinh ct \\
    &= cf(t)\sin ct - c f(t) \sin ct = 0\text{,} \\
    h'(t)
    &= f'(t)\sinh ct + cf(t)\cosh ct - c^{-1} f''(t) \cosh ct - f'(t) \sinh ct \\
    &= cf(t)\cosh ct - c^{-1} f''(t) \cosh ct \\
    &= cf(t)\cosh ct - cf(t) \cosh ct = 0\text{.}
  \end{align*} Thus $g$ and $h$ are constant. Let $a$ and $b$ be constants such
  that $g(t) = a$ and $h(t) = b$. Then \begin{align*}
    f(t)\cosh^2 ct - c^{-1} f'(t) \sinh ct\cosh ct &= a\cosh ct \text{,} \\
    f(t)\sinh^2 ct - c^{-1} f'(t) \sinh ct\cosh ct &= b\sinh ct \text{.}
  \end{align*}Subtracting,\begin{align*}
    &f(t)\cosh^2 ct - f(t)\sinh^2 ct = a\cosh ct - b\sinh ct \text{,}
  \end{align*} which simplifies to $f(t) = a\cosh ct - b\sinh ct$.
\end{proof}

\end{document}
