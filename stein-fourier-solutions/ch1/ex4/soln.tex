\documentclass[oneside]{article}

\usepackage{amsfonts}
\usepackage{amsmath}
\usepackage{amssymb}
\usepackage{amsthm}
\usepackage{enumitem}
\usepackage{mathtools}

\allowdisplaybreaks

% Notation shortcuts
\newcommand\abs[1]{\left|#1\right|}
\newcommand\defeq{\overset{\mathrm{def}}{=}}

\DeclareMathOperator{\sgn}{sgn}

\newcommand\bbC{\mathbb{C}}
\newcommand\bbR{\mathbb{R}}
\newcommand\bbZ{\mathbb{Z}}

\begin{document}
  \begin{enumerate}[label=\textbf{(\alph*)}]
    \item Define \[
      e^z = \sum_{n=0}^\infty \frac{z^n}{n!}\text{.}
    \]
    We first show that this series converges for every $z \in \bbC$.

    Define \[
      a_n = \frac{\abs{z^n}}{n!} \text.
    \] Then\[
      \frac{a_{n+1}}{a_n}
      = \frac{\abs{z^{n+1}}n!}{\abs{z^n}(n+1)!}
      = \frac{\abs{z}^{n+1}n!}{\abs{z}^n(n+1)!}
      = \frac{\abs{z}}{n+1} \text{.}
    \] Applying the ratio test,\[
      \lim_{n\to\infty} \frac{\abs{z}}{n+1} = 0\text,
    \] so the series $\sum_{n=0}^\infty a_n$ converges.

    Recalling that $\abs{z^n / n!} = \abs{z^n}/n! = a_n$, we have that \[
      e^z = \sum_{n=0}^{z^n}
    \] converges by 3(c).

    We now show that the convergence is uniform on every bounded subset of
    $\bbC$. Pick an arbitrary bounded $S \subset \bbC$ and an arbitrary
    $\epsilon > 0$. We will show that there exists an integer $M$ such that
    for all $N > M$ and $s \in S$, \begin{equation} \label{eq:uniform-conv-def}
      \abs{\sum_{n=0}^N\frac{z^n}{n!} - e^x} < \epsilon \text{.}
    \end{equation}

    Note that \eqref{eq:uniform-conv-def} is equivalent to \begin{equation}
      \label{eq:uniform-conv-def-cancelled}
      \abs{\sum_{n=N+1}^\infty\frac{z^n}{n!}} < \epsilon
    \end{equation} after cancelling the first $N$ terms of the series.

    Choose $c$ such that $c > \abs{s}$ for all $s \in S$. This is well-defined
    because $S$ is bounded. We know from above that \[
      e^c = \sum_{n=0}^\infty \frac{c^n}{n!}
    \] converges. Then there exists an integer $M$ such that for all $N > M$,\[
      \abs{\sum_{n=0}^N \frac{c^n}{n!} - e^c} < \epsilon \text{,}
    \] or after cancelling the first $N$ terms of the series,\[
      \sum_{n=N+1}^\infty \frac{c^n}{n!} < \epsilon \text{.}
    \]

    Observe that for all $n$, \[
      \frac{c^n}{n!} > \frac{\abs{z}^n}{n!} = \abs{\frac{z^n}{n!}}\text{,}
    \] so \[
      \sum_{n=N+1}^\infty \abs{\frac{z^n}{n!}} < \epsilon \text{.}
    \]

    For every partial sum from $N+1$ to some $N'$ we have\[
      \abs{\sum_{n=N+1}^{N'} \frac{z^n}{n!}}
      \leq \sum_{n=N+1}^{N'} \abs{\frac{z^n}{n!}}
    \] by the triangle inequality. Taking the limit, \begin{align*}
      \abs{\sum_{n=N+1}^\infty \frac{z^n}{n!}}
      &\leq \sum_{n=N+1}^\infty \abs{\frac{z^n}{n!}} \\
      &< \epsilon\text{,}
    \end{align*} which matches \eqref{eq:uniform-conv-def-cancelled}, concluding
    the proof.\qed

    \item We first show that the series \[
      e^z = \sum_{n=0}^\infty \frac{z^n}{n!}
    \] converges absolutely for all $z$. We have \[
      \lim_{n\to\infty}\abs{\frac{\frac{z^{n+1}}{(n+1)!}}{\frac{z^n}{n!}}}
      = \lim_{n\to\infty}\abs{\frac{z}{n+1}}
      = \lim_{n\to\infty}\frac{\abs{z}}{n+1}
      = 0\text{,}
    \] so the series converges absolutely by the ratio test.

    Observe that for the $n$\textsuperscript{th} term of the series for
    $e^{z_1+z_2}$,\begin{align*}
      \frac{(z_1+z_2)^n}{n!}
      &= \frac{1}{n!}\sum_{k=0}^n \frac{n!}{k!(n-k)!} z_1^k z_2^{n-k} \\
      &= \sum_{k=0}^n \frac{z_1^k}{k!} \frac{z_2^{n-k}}{(n-k)!} \text{.}
    \end{align*}

    We thus recognise the series for $e^{z_1 + z_2}$ as the Cauchy product of
    the series for $e^{z_1}$ and $e^{z_2}$. Since we've shown that these
    converge absolutely, $e^{z_1 + z_2} = e^{z_1}e^{z_2}$.\qed

    \item We first find the power series of $\cos y$ around $0$. We have
    \begin{align*}
      \cos0 &= \cos 0 = 1\text{,} \\
      \cos'0 &= -\sin 0 = 0\text{,} \\
      \cos''0 &= -\cos 0 = -1\text{,} \\
      \cos'''0 &= \sin 0 = 0\text{,} \\
      \cos''''0 &= \cos 0 = 1\text{,}
    \end{align*} and so on. The odd terms are zero, so we can skip them and
    write our power series as \[
      \cos y = \sum_{n=0}^\infty \frac{(-1)^ny^{2n}}{(2n)!} \text{.}
    \] To prove convergence, we have\[
      \lim_{n\to\infty}\abs{\frac{\frac{(-1)^{n+1}y^{2n+2}}{(2n+2)!}}
      {\frac{(-1)^ny^{2n}}{(2n)!}}}
      = \lim_{n\to\infty}\abs{\frac{y^2}{(2n+2)(2n+1)}} = 0 \text{,}
    \] so the power series converges absolutely by the ratio test.

    We repeat the same for $\sin y$: \begin{align*}
      \sin0 &= \sin 0 = 0\text{,} \\
      \sin'0 &= \cos 0 = 1\text{,} \\
      \sin''0 &= -\sin 0 = 0\text{,} \\
      \sin'''0 &= -\cos 0 = -1\text{,} \\
      \sin''''0 &= \sin 0 = 0\text{,}
    \end{align*} and so on. Collapsing the even terms, which are zero, we write \[
      \sin y = \sum_{n=0}^\infty \frac{(-1)^ny^{2n+1}}{(2n+1)!} \text{.}
    \] Again proving convergence, we have\[
      \lim_{n\to\infty}\abs{\frac{\frac{(-1)^{n+1}y^{2n+3}}{(2n+3)!}}
      {\frac{(-1)^ny^{2n+1}}{(2n+1)!}}}
      = \lim_{n\to\infty}\abs{\frac{y^2}{(2n+3)(2n+2)}}
      = 0\text{,}
    \] so this power series also converges absolutely by the ratio test.

    We combine the power series as \begin{align*}
      \cos y + i\sin y &= \sum_{n=0}^\infty \frac{(-1)^ny^{2n}}{(2n)!}
        + i \sum_{n=0}^\infty \frac{(-1)^ny^{2n+1}}{(2n+1)!} \\
      &= \sum_{n=0}^\infty \frac{(-1)^ny^{2n}}{(2n)!}
        + \sum_{n=0}^\infty \frac{(-1)^niy^{2n+1}}{(2n+1)!} \\
      &= \sum_{n=0}^\infty \frac{i^ny^n}{n!} \text{,}
    \end{align*} where the $n$\textsuperscript{th} term of the power series for
    $\cos y$ becomes the $2n$\textsuperscript{th} term of the combined series,
    and the $n$\textsuperscript{th} term of the series for $\cos y$ becomes the
    $2n + 1$\textsuperscript{th} term of the combined series. We are able to
    combine the two power series into one because they are absolutely
    convergent.

    Then \begin{align*}
      \cos y + i\sin y &= \sum_{n=0}^\infty \frac{i^ny^n}{n!} \\
      &= \sum_{n=0}^\infty \frac{(iy)^n}{n!} \\
      &= e^{iy}\text,
    \end{align*} where the last equality is by our definition of complex
    exponentiation.\qed

    \item Let $x, y \in \bbR$. We have \begin{align*}
      e^{x+iy} &= e^xe^{iy} & \text{(by (b))} \\
      &= e^x(\cos y + i\sin y) \text{.} & \text{(by (c))} \\
    \end{align*}

    Observe that \begin{align}
      \label{eq:eiy-abs}
      \abs{e^{iy}} &= \abs{\cos y + i\sin y} \nonumber\\
      &= \sqrt{(\cos y)^2 + (\sin y)^2} \\
      &= 1 \text{.} \nonumber
    \end{align}

    Then \begin{align*}
      \abs{e^{x + iy}} &= \abs{e^xe^{iy}} & \text{(by (b))} \\
      &= \abs{e^x}\abs{e^{iy}} & \text{(by 1(d))} \\
      &=\abs{e^x}\text{.} & \text{(by \eqref{eq:eiy-abs})}
    \end{align*}\qed

    \item \begin{itemize}[leftmargin=34pt]
      \item[($\implies$)]
      Decompose $z = x + iy$ for $x, y \in \bbR$. Then
        $e^z = e^x\cos y + ie^x \sin y$.

        We set $e^z = 1$. Equating the imaginary components,\[
          e^x\sin y = 0 \text{.}
        \] Since $e^x > 0$, $\sin y = 0$.

        Similarly equating the real components,\[
          e^x\cos y = 1 \text{.}
        \] Since $\sin y = 0$, either $\cos y = 1$ or $\cos y = -1$. We know
        that $\cos y > 0$ since $e^x > 0$ and their product is positive. Hence
        $\cos y = 1$.

        We have $\sin y = 0$ and $\cos y = 1$, so $y = 2\pi k$ for some
        $k \in \bbZ$.

        Finally $1 = e^x \cos y = e^x$, so $x = 0$.

        Thus, $z = x + iy = 2\pi k i$ for some $k \in \bbZ$.

      \item[($\impliedby$)] Let $k$ be an arbitrary integer. Let $z = 2\pi k i$.
      Then\begin{align*}
        e^z &= e^{2\pi k i} \\
        &= \cos 2\pi k + i\sin 2\pi k &\text{(by (c))} \\
        &= 1 + 0i \\
        &= 1 \text{.}
      \end{align*}\qed
      \end{itemize}

    \item Let $z = x + iy$ for some $x, y \in \bbR$. Let $r = \abs{z}$.
      Observe that $0 \leq r < \infty$.

      We show that this is the unique choice of $r \geq 0$ if we wish to
      represent\[
        z = re^{i\theta}
      \] for some $\theta \in \bbR$.

      Observe that \begin{align*}
        \abs{e^{i\theta}}
        &= \sqrt{\cos^2\theta + \sin^2\theta} \\
        &= 1\text{,}
      \end{align*} so\begin{align*}
        \abs{z} &= \abs{re^{i\theta}} \\
        &= \abs{r}\abs{e^{i\theta}} &\text{(by 1(d))} \\
        &= \abs{r} \\
        &= r\text{.} &\text{(we've restricted $r \geq 0$)}
      \end{align*}

      To pick $\theta$, show that it (along with $r$) represents $z$, and show
      its uniqueness, we argue by cases.\begin{itemize}[leftmargin=64pt]
        \item[($x = 0, y = 0$)] $r = 0$. For any choice of
        $\theta \in \bbR$, \[
          re^{i\theta} = 0e^{i\theta} = 0 \text{,}
        \] so in this degenerate case our choice of $\theta \in \bbR$ can be
        completely arbitrary.

        \item[($x = 0, y \neq 0$)] $r = \abs{y}$. We want \begin{align*}
          z &= iy \\
          &= \abs{y}e^{i\theta} \\
          &= \abs{y}(\cos\theta + i\sin\theta) \text{.}
        \end{align*}
        Equating the imaginary components, we find
        $\abs{y}\sin\theta = y$ or $\sin\theta = \sgn y$.
        Thus \[
          \theta = \frac{\pi}{2}\sgn y + 2\pi k
        \] for some $k \in \bbZ$.

        This satisfies our equation in the real components as well since
        $\cos\theta = 0$ for our choice of $\theta$, as required.

        \item[($x \neq 0$)] We want \begin{align*}
          z &= x + iy \\
          &= re^{i\theta} \\
          &= \abs{z}e^{i\theta} \\
          &= \abs{z}\cos{\theta} + i\abs{z}\sin{\theta} \text{.}
        \end{align*}
        Equating the real and imaginary sides,\begin{align*}
          \cos\theta &= \frac{x}{\sqrt{x^2 + y^2}} \text{,} \\
          \sin\theta &= \frac{y}{\sqrt{x^2 + y^2}} \text{.}
        \end{align*} Note that these two equations are sufficient and necessary
        to obtain $\theta$ that represents $z$.

        $\theta = \arctan(y/x) + 2\pi k$ for some arbitrary $k \in \bbZ$
        describes all such $\theta$.\qed

      \end{itemize}

    \item Multiplying a complex number by $i$ rotates it anticlockwise around
    the origin by $\pi/2$ radians. More generally, multipying a complex number
    by $e^{i\theta}$ rotates it anticlockwise around the origin by $\theta$
    radians.

    \item \begin{align*}
      \frac{e^{i\theta}+e^{-i\theta}}{2}
      &= \frac{\cos\theta + i\sin\theta + \cos\theta + i\sin(-\theta)}{2}
      & \text{(by (c))}\\
      &= \frac{\cos\theta + i\sin\theta + \cos\theta - i\sin\theta}{2} \\
      &= \frac{2\cos\theta}{2} \\
      &= \cos\theta
    \end{align*}
    \begin{align*}
      \frac{e^{i\theta}-e^{-i\theta}}{2i}
      &= \frac{\cos\theta + i\sin\theta - \cos\theta - i\sin(-\theta)}{2i}
      & \text{(by (c))}\\
      &= \frac{\cos\theta + i\sin\theta - \cos\theta + i\sin\theta}{2i} \\
      &= \frac{2i\sin\theta}{2i} \\
      &= \sin\theta
    \end{align*}
    \qed

    \item Using Euler's identity from (h),\begin{align*}
      &\cos\theta\cos\vartheta - \sin\theta\sin\vartheta \\
      &= \frac14
        \left(e^{i\theta} + e^{-i\theta}\right)
        \left(e^{i\vartheta} + e^{-i\vartheta}\right)
        - \frac1{4i^2}
        \left(e^{i\theta} - e^{-i\theta}\right)
        \left(e^{i\vartheta} - e^{-i\vartheta}\right) \\
      &= \frac14\left(
        \left(e^{i\theta} + e^{-i\theta}\right)
        \left(e^{i\vartheta} + e^{-i\vartheta}\right)
        + \left(e^{i\theta} - e^{-i\theta}\right)
        \left(e^{i\vartheta} - e^{-i\vartheta}\right)\right) \\
      &= \frac12\left(
        e^{i\theta}e^{i\vartheta}
        + e^{-i\theta}e^{-i\vartheta}
        \right) \\
      &= \frac12\left(
        e^{i(\theta + \vartheta)} + e^{-i(\theta + \vartheta)}
        \right)  & \text{(by (b))} \\
      &= \cos(\theta + \vartheta) \text{.} & \text{(by (h))}
    \end{align*} Swapping $\vartheta$ for $-\vartheta$ and observing that
    $\cos$ is even and $\sin$ is odd shows that \[
      \cos(\theta  - \vartheta)
      = \cos\theta\cos\vartheta + \sin\theta\sin\vartheta \text{.}
    \]

    Arguing similarly for the $\sin$ identities,\begin{align*}
      &\sin\theta\cos\vartheta + \cos\theta\sin\vartheta \\
      &= \frac1{4i}
        \left(e^{i\theta} - e^{-i\theta}\right)
        \left(e^{i\vartheta} + e^{-i\vartheta}\right)
        + \frac1{4i}
        \left(e^{i\theta} + e^{-i\theta}\right)
        \left(e^{i\vartheta} - e^{-i\vartheta}\right) \\
      &= \frac1{4i}\left(
        \left(e^{i\theta} - e^{-i\theta}\right)
        \left(e^{i\vartheta} + e^{-i\vartheta}\right)
        + \left(e^{i\theta} + e^{-i\theta}\right)
        \left(e^{i\vartheta} - e^{-i\vartheta}\right)\right) \\
      &= \frac1{2i}\left(
        e^{i\theta}e^{i\vartheta}
        - e^{-i\theta}e^{-i\vartheta}
        \right) \\
      &= \frac1{2i}\left(
        e^{i(\theta + \vartheta)} - e^{-i(\theta + \vartheta)}
        \right)  & \text{(by (b))} \\
      &= \sin(\theta + \vartheta) \text{.} & \text{(by (h))}
    \end{align*} Again swapping $\vartheta$ for $-\vartheta$ shows that \[
      \sin(\theta  - \vartheta)
      = \sin\theta\cos\vartheta - \cos\theta\sin\vartheta \text{.}
    \]

    We list the identities:\begin{align}
      \cos(\theta + \vartheta)
        &= \cos\theta\cos\vartheta
        - \sin\theta\sin\vartheta\text{,} \label{eq:cosplus} \\
      \cos(\theta - \vartheta)
        &= \cos\theta\cos\vartheta
        + \sin\theta\sin\vartheta\text{,} \label{eq:cosminus} \\
      \sin(\theta + \vartheta)
        &= \sin\theta\cos\vartheta
        + \cos\theta\sin\vartheta\text{,} \label{eq:sinplus} \\
      \sin(\theta - \vartheta)
        &= \sin\theta\cos\vartheta
        - \cos\theta\sin\vartheta\text{.} \label{eq:sinminus}
    \end{align}

    Subtracting the LHS and RHS of \eqref{eq:cosplus} from \eqref{eq:cosminus},
    \[
      2\sin\theta\sin\vartheta
      = \cos(\theta - \vartheta) - \cos(\theta + \vartheta) \text{.}
    \] Similarly adding \eqref{eq:sinplus} and \eqref{eq:sinminus},\[
      2\sin\theta\cos\vartheta
      = \sin(\theta + \vartheta) + \sin(\theta - \vartheta) \text{.}
    \]\qed

  \end{enumerate}

\end{document}
