\documentclass[oneside]{article}

\usepackage{amsfonts}
\usepackage{amsmath}
\usepackage{amssymb}
\usepackage{amsthm}
\usepackage{enumitem}
\usepackage{mathtools}

\allowdisplaybreaks

% Notation shortcuts
\newcommand\abs[1]{\left|#1\right|}
\newcommand\defeq{\overset{\mathrm{def}}{=}}
\newcommand*\Laplace{\mathop{}\!\mathbin\bigtriangleup}

\DeclareMathOperator{\sgn}{sgn}
\DeclareMathOperator{\atantwo}{atan2}

\newcommand\bbC{\mathbb{C}}
\newcommand\bbR{\mathbb{R}}
\newcommand\bbZ{\mathbb{Z}}

\begin{document}
Let $F$ be a function on $(a, b)$ with two continuous derivatives.

By Taylor's theorem, \[
  F'(y) = F'(x) + (y-x)F''(x) + (y-x)\eta(x)
\] with $\lim_{x\to y}\eta(x) = 0$. Setting \[
  \psi(x) = \eta(y-x)
\] we get \[
  F'(y) = F'(x) + (y-x)F''(x) + (y-x)\psi(y-x)
\] with $\lim_{h\to0}\psi(h) = 0$.

Then \begin{align*}
  &F(x+h) - F(x) \\
  &= \int_x^{x+h}F'(y)dy \\
  &= \int_x^{x+h}F'(x)dy + \int_x^{x+h}(y-x)F''(x)dy + \int_x^{x+h}(y-x)\psi(y-x)dy \\
  &= hF'(x) + \frac{h^2}{2}F''(x) + h^2\varphi(h) \text{,}
\end{align*}
where in the last line we use\begin{align*}
  \int_x^{x+h}(y-x)\psi(y-x)dy
  &= \int_0^ht\psi(t)dt \\
  &= \psi(\eta)\int_0^htdt \\
  &= \frac{h^2}{2}\psi(\eta)
\end{align*} for some $\eta$ between $0$ and $h$ and set
$\varphi(h) = \psi(\eta) / 2$. Then $\varphi(h) \to 0$ as $h \to 0$.

Hence,\[
  F(x+h) = F(x) + hF'(x) + \frac{h^2}{2}F''(x) + h^2\varphi(h)
\] with $\lim_{h\to0}\varphi(h) = 0$.

Hence\begin{align*}
  &F(x+h) + F(x-h) - 2F(x) \\
  &= F(x) + hF'(x) + \frac{h^2}{2}F''(x) + h^2\varphi(h) \\
  &\qquad+ F(x) - hF'(x) + \frac{h^2}{2}F''(x) + h^2\varphi(-h) \\
  &\qquad- 2F(x) \\
  &= h^2F''(x) + h^2\varphi(h) + h^2\varphi(-h) \text{.}
\end{align*} Then \begin{align*}
  \lim_{h\to0}\frac{F(x+h) + F(x-h) - 2F(x)}{h^2}
  &= \lim_{h\to0}\left(F''(x) + \varphi(h) + \varphi(-h)\right) \\
  &= F''(x)\text{,}
\end{align*} where we use the fact that $\varphi(h) \to 0$ as $h \to 0$.\qed

\end{document}
