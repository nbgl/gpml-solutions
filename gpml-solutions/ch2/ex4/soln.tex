\documentclass{article}

\usepackage{amsmath}
\usepackage{amsthm}

\DeclareMathOperator{\var}{var}

\newcommand{\bfk}{\mathbf{k}}
\newcommand{\bfx}{\mathbf{x}}

\begin{document}
  Let $X$ be a matrix of $n-1$ observations. Let $\bfx_\circ$ be an additional
  observation. Let $\bfx_*$ be an arbitrary test point.

  Define \begin{align*}
    K &= k(X, X) \text{,} \\
    \bfk_\circ &= k(X, \bfx_\circ) \text{,} \\
    k_{\circ\circ} &= k(\bfx_\circ, \bfx_\circ) \text{,} \\
    \bfk_* &= k(X, \bfx_*) \text{,} \\
    k_{\circ*} &= k(\bfx_\circ, \bfx_*) \text{,} \\
    k_{**} &= k(\bfx_*, \bfx_*) \text{.}
  \end{align*}

  Then setting $M = K + \sigma^2_n I$ and
  $c = k_{\circ\circ} + \sigma^2_n - \bfk_\circ^\top M^{-1}\bfk_\circ$,
  \begin{align*}
    &\var_{n-1}(f(\bfx_*)) \\
    %
    &= k_{**} - \bfk_*^\top
    \left(K + \sigma^2_n I \right)^{-1} \bfk_* \text{,} \\
    %
    &= k_{**} - \bfk_*^\top M^{-1} \bfk_* \text{,} \\
    %
    \\
    %
    &\var_n(f(\bfx_*)) \\
    %
    &= k_{**} - \begin{bmatrix} \bfk_*^\top & k_{\circ*} \end{bmatrix}
    \left(\begin{bmatrix}
    K & \bfk_\circ \\ \bfk_\circ^\top & k_{\circ\circ}
    \end{bmatrix} + \sigma^2_n I \right)^{-1}
    \begin{bmatrix} \bfk_* \\ k_{\circ*} \end{bmatrix} \\
    %
    &= k_{**} - \begin{bmatrix} \bfk_*^\top & k_{\circ*} \end{bmatrix}
    \begin{bmatrix}
    K + \sigma^2_n I & \bfk_\circ \\
    \bfk_\circ^\top & k_{\circ\circ} + \sigma^2_n
    \end{bmatrix}^{-1}
    \begin{bmatrix} \bfk_* \\ k_{\circ*} \end{bmatrix} \\
    %
    &= k_{**} - \begin{bmatrix} \bfk_*^\top & k_{\circ*} \end{bmatrix}
    \begin{bmatrix}
    M^{-1}+ c^{-1}M^{-1}\bfk_\circ\bfk_\circ^\top M^{-1} &
    -c^{-1}M^{-1} \bfk_\circ \\
    -c^{-1}\bfk_\circ^\top M^{-1} &
    c^{-1}
    \end{bmatrix}
    \begin{bmatrix} \bfk_* \\ k_{\circ*} \end{bmatrix} \text{,}
  \end{align*} where we perform inversion using an identity from section A.3.
  Then \begin{align*}
    &\var_{n-1}(f(\bfx_*)) - \var_n(f(\bfx_*)) \\
    &= c^{-1}\bfk_*^\top M^{-1}\bfk_\circ\bfk_\circ^\top M^{-1} \bfk_*
    - c^{-1} \bfk_*^\top M^{-1} \bfk_\circ
    - c^{-1} \bfk_\circ^\top M^{-1} \bfk_*
    + c^{-1} \\
    &= \frac{\left(\bfk_\circ^\top M^{-1}\bfk_*\right)^2
             - 2\bfk_\circ^\top M^{-1}\bfk_*
             + 1}{c} \\
    &= \frac{\left(1 - \bfk_\circ^\top M^{-1}\bfk_*\right)^2}{c} \\
    &= \frac{\left(1 - \bfk_\circ^\top \left(K + \sigma^2_n I\right)^{-1}
    \bfk_*\right)^2}{
    k_{\circ\circ} + \sigma^2_n
    - \bfk_\circ^\top \left(K + \sigma^2_n I\right)^{-1}\bfk_\circ} \text{.}
  \end{align*} The numerator is the square of a real number, so it is
  non-negative. The denominator is the Schur complement of \[
    k\left(\begin{bmatrix}X&\bfx_\circ\end{bmatrix},
    \begin{bmatrix}X&\bfx_\circ\end{bmatrix}\right) + \sigma_n^2 I
    = \begin{bmatrix}
    K + \sigma^2_n I & \bfk_\circ \\
    \bfk_\circ^\top & k_{\circ\circ} + \sigma^2_n
    \end{bmatrix} \text{.}
  \] By assumption $k$ is positive-definite and $\sigma_n^2 > 0$, so this matrix
  is also positive definite as the sum of positive-definite matrices. Then its
  Schur complement is positive by Boyd and Vandenberghe, section A.5.5.

  Hence, $\var_{n-1}(f(\bfx_*)) - \var_n(f(\bfx_*)) \geq 0$ and \[
      \var_{n-1}(f(\bfx_*)) \geq \var_n(f(\bfx_*)) \text{.}
  \]\qed


\end{document}
